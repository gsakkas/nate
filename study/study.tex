\documentclass{article}

\usepackage{listings}
\lstset{
  basicstyle=\ttfamily,
  keywordstyle=\ttfamily,
  escapeinside={(*@}{@*)},
}

\usepackage{xcolor}
\newcommand\hlBlame[1]{\colorbox{red!25}{#1}}

\begin{document}


\section{Debugging and Functional Programming}

You will be shown OCaml programs that \emph{do not type-check}, the code
implicated by the type checker will be \hlBlame{highlighted}. You will
also be given English comments explaining what the program \emph{should}
do, as well as assertions that \emph{should} type check \emph{and} succeed.


\subsection{Concatenating Strings}

\subsubsection{Control}
\begin{lstlisting}
  (* "sepConcat sep [s1;s2;s3]" should insert "sep"
     between "s1", "s2", and "s3", and concatentate
     the result. *)

  let rec sepConcat sep sl =
    match sl with
    | [] -> ""
    | h::t ->
        let f a x = a ^ (sep ^ x) in
        let base = [] in
        (*@\hlBlame{List.fold\_left f base sl}@*)

  assert( sepConcat "," ["foo"; "bar"; "baz"] = "foo,bar,baz" )
\end{lstlisting}

\begin{enumerate}
\item Why is \verb!sepConcat! not well-typed?
\item Fix the \verb!sepConcat! program.
\end{enumerate}

\subsubsection{Treatment}
\begin{lstlisting}
  (* "sepConcat sep [s1;s2;s3]" should insert "sep"
     between "s1", "s2", and "s3", and concatentate
     the result. *)

  let rec sepConcat sep sl =
    match sl with
    | [] -> ""
    | h::t ->
        let f a x = a ^ (sep ^ x) in
        let base = (*@\hlBlame{[]}@*) in
        List.fold_left f base sl

  assert( sepConcat "," ["foo"; "bar"; "baz"] = "foo,bar,baz" )
\end{lstlisting}

\begin{enumerate}
\item Why is \verb!sepConcat! not well-typed?
\item Fix the \verb!sepConcat! program.
\end{enumerate}

\subsection{Padding Lists}

\subsubsection{Control}
\begin{lstlisting}
  (* "padZero xs ys" returns a pair "(xs', ys')"
     where the shorter of "xs" and "ys" has been
     left-padded by zeros until both lists have
     equal length. *)

  let rec clone x n =
    if n <= 0 then
      []
    else
      x :: clone x (n - 1)

  let padZero l1 l2 =
    let n = List.length l1 - List.length l2 in
    if n < 0 then
      (clone 0 ((-1) * n) @ l2, l2)
    else
      (l1, (*@\hlBlame{clone 0 n}@*) :: l2)

  assert( padZero [1;2] [1] = ([1;2], [0;1]) )
\end{lstlisting}

\subsubsection{Treatment}
\begin{lstlisting}
  (* "padZero xs ys" returns a pair "(xs', ys')"
     where the shorter of "xs" and "ys" has been
     left-padded by zeros until both lists have
     equal length. *)

  let rec clone x n =
    if n <= 0 then
      []
    else
      x :: clone x (n - 1)

  let padZero l1 l2 =
    let n = List.length l1 - List.length l2 in
    if n < 0 then
      (clone 0 ((-1) * n) @ l2, l2)
    else
      (l1, clone 0 n (*@\hlBlame{::}@*) l2)

  assert( padZero [1;2] [1] = ([1;2], [0;1]) )
\end{lstlisting}

\begin{enumerate}
\item Why is \verb!padZero! not well-typed?
\item Fix the \verb!padZero! program.
\end{enumerate}

\subsection{Big-Integer Arithmetic}

\subsubsection{Control}
\begin{lstlisting}
  (* "mulByDigit d [n1;n2;n3]" should multiply the
     "big integer" "[n1;n2;n3]"
     by the single digit "d". *)

  let rec mulByDigit d n =
    match List.rev n with
    | []   -> []
    | h::t -> [(*@\hlBlame{mulByDigit d t}@*); (h * d) mod 10]

  assert( mulByDigit 4 [2;5] = [1;0;0] )
\end{lstlisting}

\begin{enumerate}
\item Why is \verb!mulByDigit! not well-typed?
\item Fix the \verb!mulByDigit! program.
\end{enumerate}

\subsubsection{Treatment}
\begin{lstlisting}
  (* "mulByDigit d [n1;n2;n3]" should multiply the
     "big integer" "[n1;n2;n3]"
     by the single digit "d". *)

  let rec mulByDigit d n =
    match List.rev n with
    | []   -> []
    | h::t -> (*@\hlBlame{[mulByDigit d t; (h * d) mod 10]}@*)

  assert( mulByDigit 4 [2;5] = [1;0;0] )
\end{lstlisting}

\begin{enumerate}
\item Why is \verb!mulByDigit! not well-typed?
\item Fix the \verb!mulByDigit! program.
\end{enumerate}

\end{document}
