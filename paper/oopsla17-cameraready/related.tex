\mysection{Related Work}
\label{sec:related-work}
\label{sec:type-error-diagnosis}

In this section we describe two relevant aspects of related work:
%
programming languages approaches to diagnosing type errors, and
%
software engineering approaches to fault localization.
%
% applying machine learning techniques to problems in programming languages.
%
% \mysubsection{Type Error Diagnosis}
%% Languages with static type systems and type inference produce type
%% errors that novices often perceive as difficult to
%% interpret~\citep{Wand1986-nw}.
%% %
%% For example, approaches based on Hindley-Milner or other constraint
%% systems typically have the issue that when the constraint $\alpha=\beta$
%% is violated, the system cannot know whether the user should intend to
%% fix $\alpha$ or $\beta$ and must thus report the discrepancy in a
%% generic manner.
%% %
%% As a result, a number of approaches have been proposed to
%% more precisely localize the type error,
%% give clearer error messages, or
%% fix the error automatically.
%
% The technique most related to our work is \ES{TODO}.  It has been shown
% to be quite good at \ES{TODO}. However, we choose to focus on \ES{TODO},
% and our approach contains \ES{TODO}, which is not present in previous
% work.

\mypara{Localizing Type Errors}
It is well-known that the original Damas-Milner
algorithm $\mathcal{W}$ produces errors far
from their source, that novices percieve as
difficult to interpret~\citep{Wand1986-nw}.
%
%% The type checker collects typing constraints as it traverses the
%% program, and it crucially assumes that if a constraint can be safely
%% added to the set of assumptions (\ie no type error has been found yet),
%% then the constraint can be \emph{assumed} to be correct.
%
The type checker reports an error the moment
it finds a constraint that contradicts one
of the assumptions, blaming the new inconsistent
constraint, and thus it is extremely sensitive
to the order in which it traverses the source
program (the infamous ``left-to-right''
bias~\citep{McAdam1998-ub}).
%
Several alternative traversal have been proposed,
\eg top-down rather than bottom-up~\citep{Lee1998-ys},
or a \emph{symmetric} traversal that checks
sub-expressions independently and only reports an
error when two inconsistent sets of constraints are
merged~\citep{McAdam1998-ub,Yang1999-yr}.
%
%\mypara{Slicing}
Type error \emph{slicing}~\citep{Haack2003-vc,Tip2001-qp,Rahli2010-ps}
overcomes the constraint-order bias by extracting a
complete and minimal subset
of terms that contribute to the error, \ie all of the
terms that are required for it to manifest and no more.
%
Slicing typically requires rewriting the type checker with a
specialized constraint language and solver, though
\citet{Schilling2011-yf} shows how to turn any type checker into a
slicer by treating it as a black-box.
%
While slicing techniques guarantee enough information to diagnose the
error, they can fall into the trap of providing \emph{too much}
information, producing a slice that is not much smaller than
the input. %original. % input program.

\mypara{Finding Likely Errors}
Thus, recent work has focused on finding the \emph{most likely} source
of a type error.
%
\citet{Zhang2014-lv} use Bayesian reasoning to search the constraint
graph for constraints that participate in many unsatisfiable paths and
relatively few satisfiable paths, based on the intuition that the
program should be mostly correct.
%
\citet{Pavlinovic2014-mr} translate the localization problem into a
MaxSMT problem, asking an off-the-shelf solver to find the smallest
set of constraints that can be removed such that the resulting system is
satisfiable.
%
\citet{Loncaric2016-uk} improve the scalability of
\citeauthor{Pavlinovic2014-mr} by reusing the existing type checker as
a theory solver in the Nelson-Oppen~\citeyear{Nelson1979-td}
style, and thus require only a MaxSAT solver.
%
All three of these techniques support \emph{weighted} constraints to
incorporate knowledge about the frequency of different errors,
but only \citeauthor{Pavlinovic2014-mr} use the weights, setting them to
the size of the term that induced the constraint.
%
In contrast, our classifiers learn a set of heuristics for predicting
the source of type errors by observing a set of ill-typed programs and
their subsequent fixes, in a sense using \emph{only} the weights and no
constraint solver.
%
It may be profitable to combine both approaches, \ie learn a set of good
weights for one of the above techniques from our training data.

\mypara{Explaining Type Errors}
In this paper we have focused solely on the task of \emph{localizing} a
type error, but a good error report should also \emph{explain} the
error.
%
\citet{Wand1986-nw}, \citet{Beaven1993-hb}, and \citet{Duggan1996-by}
attempt to explain type errors by collecting the chain of inferences
made by the type checker %--- essentially the typing derivation ---
and presenting them to the user.
%
% Such explanations can be lengthy, as an attempt to compress the
% explanation \citet{Yang2000-kz} presents a visualization of the
% inference process.
%
\citet{Gast2004-zd} produces a slice enhanced by arrows
showing the dataflow from sources with different types to a
shared sink, borrowing the insight of dataflows-as-explanations from
\textsc{MrSpidey}~\citep{Flanagan1996-bu}.
%
\citet{Hage2006-hc} catalog a set of heuristics for
improving the quality of error messages by examining errors made by
novices.
%
\citet{Heeren2003-db}, \citet{Christiansen2014-qc}, and
\citet{Serrano2016-oo} extend the ability to customize error messages to
library authors, enabling \emph{domain-specific} errors.
%
%\mypara{Interactive Explanations}
Such \emph{static} explanations of type errors run the risk of
overwhelming the user with too much information, it may be preferable to
treat type error diagnosis as an \emph{interactive} debugging session.
%
\citet{Bernstein1995-yj} extend the type inference procedure to handle
\emph{open} expressions (\ie with unbound variables), allowing users to
interactively query the type checker for the types of sub-expressions.
%
\citet{Chitil2001-td} proposes \emph{algorithmic debugging} of type
errors, presenting the user with a sequence of yes-or-no questions about
the inferred types of sub-expressions that guide the user to a specific
explanation.
%
\citet{Seidel2016-ul} explain type errors by searching for inputs that
expose the \emph{run-time} error that the type system prevented, and
present users with an interactive visualization of the erroneous
computation.

%% RJ-CUT-ME \mypara{Programmatic Explanations}
%% RJ-CUT-ME %
%% RJ-CUT-ME The best explanation of a type error, however, might be given by an
%% RJ-CUT-ME expert, \eg the compiler or library author.
%% RJ-CUT-ME %
%% RJ-CUT-ME \citet{Hage2006-hc} catalog a set of heuristics for
%% RJ-CUT-ME improving the quality of error messages by examining errors made by
%% RJ-CUT-ME novices.
%% RJ-CUT-ME %
%% RJ-CUT-ME \citet{Heeren2003-db}, \citet{Christiansen2014-qc}, and
%% RJ-CUT-ME \citet{Serrano2016-oo} extend the ability to customize error messages to
%% RJ-CUT-ME library authors, enabling \emph{domain-specific} errors.
%
% The 8.0 release of the
% Glasgow Haskell Compiler\footnote{\url{https://ghc.haskell.org/trac/ghc/wiki/Proposal/CustomTypeErrors}}
% incorporates these ideas, allowing library authors to supply
% custom errors when type-class resolution or type-family reduction fail,
% but not for ordinary unification failures.


\mypara{Fixing Type Errors}
Some techniques go beyond explaining or locating a type error,
and actually attempt to \emph{fix} the error automatically.
%
\citet{Lerner2007-dt} searches for fixes by enumerating a
set of local mutations to the program and querying the type checker to
see if the error remains.
%
\citet{Chen2014-gd} use a notion of \emph{variation-based} typing to
track choices made by the type checker and enumerate potential
type (and expression)
changes that would fix the error.
%
They also extend the algorithmic debugging technique of
\citeauthor{Chitil2001-td} by allowing the user to enter the expected
type of specific sub-expressions and suggesting fixes based on these
desired types \citeyear{Chen2014-vm}.
%
Our classifiers do not attempt to suggest fixes to type errors, but it
may be possible to do so by training a classifier to predict the
syntactic class of each expression in the \emph{fixed} program --- we
believe this is an exciting direction for future work.

\mypara{Fault Localization}
%
Given a defect, \emph{fault localization} is the task of identifying
``suspicious'' program elements (\eg lines, statements) that are likely
implicated in the defect %(\ie that should be changed to fix the defect)
%
--- thus, type error localization can be viewed as an instance of fault
localization.
%
The best-known fault localization technique is likely Tarantula, which
uses a simple mathematical formula based on measured information from
dynamic normal and buggy runs~\citep{Jones2002-us}.
%
Other similar approaches, including those of \citet{Chen2002-qz} and
\citet{Abreu2006-fn,Abreu2007-mu} consider alternate features of
information or refined formulae and generally obtain more precise
results; see \citet{Wong2009-pd} for a survey.
%
While some researchers have approached such fault localization with an
eye toward optimality (\eg \citet{Yoo2013-rw} determine optimal
coefficients), in general such fault localization approaches are limited
by their reliance on either running tests or including relevant
features.
%
For example, Tarantula-based techniques require a normal and a buggy run
of the program.
%
By contrast, we consider incomplete programs with type errors that may
not be executed in any standard sense.
%
Similarly, the features available influence the classes of defects that
can be localized.
%
For example, a fault localization scheme based purely on control flow features
will have difficulty with cross-site scripting or SQL code injection
attacks, which follow the same control flow path on normal and buggy
runs (differing only in the user-supplied data).
%
Our feature set is comprised entirely of syntactic and typing features,
a natural choice for type errors, but it would likely not
generalize to other defects.


% \subsection{Machine Learning for Programming Languages}
% \label{sec:ml-pl}



%%% Local Variables:
%%% mode: latex
%%% TeX-master: "main"
%%% End:
