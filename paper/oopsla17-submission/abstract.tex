
\begin{abstract}
Localizing type errors is challenging in
languages with global type inference, as
the type checker must make assumptions
about what the programmer intended to do.
%
We introduce \toolname, a \emph{data-driven}
approach to error localization based on
supervised learning.
%
\toolname analyzes a large corpus
of training data --- pairs of ill-typed
programs and their ``fixed'' versions ---
to automatically \emph{learn a model}
of where the error is most likely
to be found.
%
Given a new ill-typed program,
\toolname executes the model to
generate a list of potential blame
assignments ranked by likelihood.
%
We evaluate \toolname by comparing its
precision to the state of the art
on a set of over 4,500 ill-typed \ocaml
programs drawn from introductory
programming classes.
%
We show that when the top-ranked blame assignment
is considered, \toolname's data-driven
model is able to correctly predict
the exact sub-expression that should
be changed \HiddenFhTopOne\% of the time, %which is
\ToolnameWinOcaml points higher than \ocaml and
\ToolnameWinSherrloc points higher than the state-of-the-art
\sherrloc tool.
%
Furthermore, \toolname's accuracy surpasses
\HiddenFhTopTwo\% when we consider the top \emph{two}
locations and reaches \HiddenFhTopThree\% if we consider
the top \emph{three}.
\end{abstract}
