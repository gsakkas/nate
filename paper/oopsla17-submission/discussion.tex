\mysection{Limitations}
\label{sec:discussion}

We have shown that we can outperform the state of the art in type error
localization by learning a model of the errors that programmers make,
using a set of features that closely resemble the information the type
checker sees.
%
% However, there is certainly room for improvement.
%
In this section we highlight some limitations of our approach and
potential avenues for future work.

% \mysubsection{Limitations}
% \label{sec:limitations}

\mypara{User-Defined Types}
Probably the single biggest limitation of our technique is that we have
(a finite set of) features for specific data and type constructors.
%
Anything our models learn about errors made with the |::| constructor or
the |list| type cannot easily be translated to new, user-defined
datatypes the model has never encountered.
%
We can mitigate this, to some extent, by adding generic
syntactic features for data constructors and match expressions, but it
remains to be seen how much these help. % \ES{TODO: UW data!}.
%
Furthermore, there is no obvious analog for transferring knowledge to
new type constructors, which we have seen are both more compact and
helpful.

As an alternative to encoding information about \emph{specific}
constructors, we might use a more abstract representation.
%
For example, instead of modeling |x :: 2| as a |::| constructor with a
right child of type |int|, we might model it as some (unknown) constructor
whose right child has an incompatible type.
%
We might symmetrically model the |2| as an integer literal whose type is
incompatible with its parent.
%
Anything we learn about |::| and |2| can now be transferred directly to
yet unseen types, but we run the risk generalizing \emph{too much} ---
\ie perhaps programmers make different mistakes with |list|s than they
do with other types, and are thus likely to choose different fixes.
%
Balancing the trade-off between specificity and generalizability appears
to be a challenging task.

% \begin{itemize}
% \item Hard to support in general, as each data and type constructor gets
%   its own set of syntactic features.
% \item Attempt to mitigate by adding generic ``is-datacon'' and
%   ``is-match'' feature, unclear how much it will help
% \end{itemize}

\mypara{Additional Features}
There are a number of other features that could improve the model's
ability to localize errors, that would be easier to add than
user-defined types.
%
For example, each occurrence of a variable knows only its type and its
immediate neighbors, but it may be helpful to know
about \emph{other} occurrences of the same variable.
%
If a variable is generally used as a |float| but has a
single use as an |int|, it seems likely that the
latter occurrence (or context) is to blame.
%
Similarly, arguments to a function application are not aware of the
constraints imposed on them by the function (and vice versa),
they only know that they are occurring in the context of an application.
%and the type of the application itself.
%
Finally, \emph{n-grams} on the token stream have proven effective for
probabilistic modeling of programming languages
\citep{Hindle2012-hf,Gabel2010-el}, we may find that they aid in
our task as well.
%
For example, if the observed tokens in an expression diverge from the
n-gram model's predictions, that indicates that there is something
unusual about the program at that point, and it may signal an error.

% \begin{itemize}
% \item def-use chains
% \item connect function and arguments in application (right now each only know about the application itself)
% \item ??
% \end{itemize}

\mypara{Independent vs Joint Predictions}
We treat each sub-expression as if it exists in a vacuum, but in reality
the program has a rich \emph{graphical} structure, particularly if one adds
edges connecting different occurrences of the same variable.
%
\citet{Raychev2015-jg} have used these richer models to great effect to
make \emph{interdependent} predictions about programs, \eg
de-obfuscating variable names or even inferring types.
%
One could even view our task of locating the source of an error as simply
another property to be predicted over a graphical model of the program.
%
One of the key advantages of a graphical model is that the predictions
made for one node can influence the predictions made for another node,
this is known as \emph{structured learning}.
%
For example, if, given the expression |1 + true|, we predict |true| to
be erroneous, we may be much less likely to predict |+| as erroneous.
%
We compensate somewhat for our lack of structure by adding contextual
features and by ranking our predictions by ``confidence'', but it would
be interesting to see how structured learning over graphical models
would perform.


%%% Local Variables:
%%% mode: latex
%%% TeX-master: "main"
%%% End:
