\section{Introduction}
\label{sec:introduction}

\subsection{Contributions}
\label{sec:contributions}
Thus, we identify the following novel contributions of our work.
%
\begin{enumerate}
\item An experimental evaluation of state-of-the-art type-error
  localization techniques based not on expert \emph{opinion}, but
  on the eventual \emph{fix} implemented by a novice user.
  %
  \ES{need to explain \emph{why} this is superior..}
\item A machine learning approach to \emph{modeling} (novice) type
  errors that can outperform the state-of-the-art by 10--30\% in
  localizing errors.
  %
  Our model can be trained on a relatively small amount of data --- we
  used a single class of around 50 students --- and appears to
  generalize to other instances of the same class.
  %
  \ES{and hopefully completely separate classes!}
  %
  In contrast to many approaches to \emph{fault localization} from the
  software engineering community, our model does not use any linguistic
  modeling techniques, \eg \emph{n-grams} over the token stream; rather,
  it relies entirely on features of the abstract syntax tree and a
  partial typing derivation.
  %
  \ES{if there's time, investigate adding Wes' n-gram model}
\end{enumerate}

%%% Local Variables:
%%% mode: latex
%%% TeX-master: "main"
%%% End:
