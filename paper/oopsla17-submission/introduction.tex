\section{Introduction}
\label{sec:introduction}

Type systems are one of the most widely-used techniques for program
verification. A type checker only accepts a program if it can
\emph{prove} the absence of an entire class of bugs.
%
Typed functional languages like \ocaml and \haskell are equipped with
particularly powerful type checkers that can operate without a single
annotation, alleviating a prime concern with verification techniques.
%
Yet this power comes with a cost: since type annotations signify the
programmer's \emph{intent}, without them the checker must
make assumptions about what the programmer intended to do.
%
This results in the common phenomenon of type errors being reported far
from their source, an annoyance to veterans of the type system, but a
major hurdle for novices \citep{Wand1986-nw,Joosten1993-yx}.

\begin{figure}[ht]
\begin{minipage}{0.45\linewidth}
\begin{ecode}
  let rec sumList xs =
    match xs with
    | []   -> __[]__
    | h::t -> h __+__ ==sumList t==
\end{ecode}
\end{minipage}
\begin{minipage}{0.49\linewidth}
\begin{verbatim}
File "sumList.ml", line 4, characters 16-25:
  This expression has type 'a list
  but an expression was expected of type int
\end{verbatim}
\end{minipage}
\caption{(left) An ill-typed \ocaml program that should sum the elements of a
  list. The \underline{underlined} expressions are equally valid
  locations to blame. The expression blamed by the \ocaml compiler
  is in \textbf{bold}.
  %
  % FIXME: This bolding is ambiguous, because ``let rec'', ``match'' and
  % ``with'' are also bolded (by \\ecode)! You need to find another way to
  % highlight what ocaml is yelling about.
  %
  (right) The error reported by \ocaml.}
\label{fig:sumList}
\end{figure}
%
Consider the |sumList| program in \autoref{fig:sumList}, written by a
student in an undergraduate Programming Languages course.
%
The program is meant to sum the integers in a list, but the student has
accidentally given the empty list as the base case, rather than 0.
%
The \ocaml compiler collects typing constraints as it traverses the
program, and reports an error the moment it finds an inconsistent
constraint.
%
In this case it blames the recursive call to |sumList|, complaining that
|sumList| returns a |list| where an |int| was expected by the |+|
operator. Note that this blame is inconsistent with the programmer's
intention and thus may not be helpful debugging information to a novice.

Recent work on type error \emph{localization}
%
\citep[\eg][]{Zhang2014-lv,Loncaric2016-uk,Pavlinovic2014-mr},
%
determining which expression(s) to blame for the error,
frames localization as an optimization problem, seeking a set of
expressions to blame that minimize some cost function (\eg the number of
constraints in the blame set, or the ratio of ``good'' paths through the
constraint graph to ``bad'' paths).
%
However, even these techniques fail to distinguish between the two
underlined expressions in \autoref{fig:sumList}, it would be equally
valid to blame \emph{either} of them alone.
%
The |[]| on line 3 could be changed to |0|, or the |+| on line 4 could
be changed to either |@| (list append) or |::|, all of which would give
type-correct programs.
%
Thus, these state-of-the-art techniques are forced to either blame
\emph{both} locations, or choose one \emph{arbitrarily}.

It may appear obvious to the reader that |[]| is the correct expression
to blame, but how is the type checker to know that?
%
Our solution is to \emph{learn} how to localize type errors by observing
a large set of (novice) errors and their \emph{subsequent fixes}.
%
We accomplish this via three concrete contributions.

\paragraph{\textbf{(1) A Dataset of Novice \ocaml Programs}}
Our first contribution is a large dataset of novice \ocaml programs,
including, but not limited to, \emph{ill-typed} programs.
%
We used an instrumented version of the \ocaml compiler to collect
student interactions over two instances of an undergraduate Programming
Languages course at
%
\begin{anonsuppress}
UC San Diego (IRB \#140608).
\end{anonsuppress}
\begin{noanonsuppress}
AUTHOR's INSTITUTION (IRB HIDDEN).
\end{noanonsuppress}
%
The data forms a time-series of programs submitted to the \ocaml
compiler, from which we extracted a set of \emph{ill-typed} programs
and their \emph{subsequent fixes}.
%
Our dataset comprises over 4,500 such pairs of programs, and many more
programs that are either well-typed or lack a corresponding fix.
%
The entire dataset (both the raw time-series and our extracted pairs of
ill-typed programs and fixes) is available at
%
\begin{anonsuppress}
\ES{TODO: finally post the data somehwere..}.
\end{anonsuppress}
\begin{noanonsuppress}
HIDDEN FOR DOUBLE-BLIND REVIEW.
\end{noanonsuppress}
%

\paragraph{\textbf{(2) Data-Driven Diagnosis of Type Errors}}
Our second contribution is a \emph{data-driven} technique for localizing
type errors (\autoref{sec:learning}).
%
We use machine learning techniques to train a set of classifiers that can
predict, given an ill-typed program, which expressions should be changed
for the program to type check.
%
Our classifiers can be trained on a relatively small set of programs
(\ie a single course's worth). 
% WRW notes that we say this below in contribution (3), which is only three
% sentences away, so to a first-time reader it sounds like we are
% double-counting:
%
% and outperform the state of the art by
% 15--30\% in blame accuracy.
%
The key technical challenge that we tackle is defining a suitable
interface between the type checker and the classifier, specifically
\emph{embedding} programs as feature vectors and presenting the
classifier's predictions to the user.

\paragraph{\textbf{(3) A Large-Scale Evaluation of Type Error Localization}}
Our third contribution is an evaluation of our classifiers and the
state-of-the-art techniques on our suite of over 4,500 student programs
(\autoref{sec:evaluation}), the largest evaluation of type error
localization techniques we are aware of.
%
We judge the correctness of a localization by comparing it to the
student's eventual fix and find that our classifiers are 15--30\% more
accurate than the state of the art.
%
We also examine which features contribute the most to our
classifiers' predictions, and endeavor to \emph{explain} specific
predictions that our classifiers get right (and wrong).

Overall, our results show that data-driven diagnosis is a practical
technique for localizing type errors by learning from past mistakes.
%
% We believe, furthermore, that with the right set of features our
% approach could be applied to other domains where there are multiple,
% equally valid explanations for an error.
% \ES{this sounds really dumb... need better phrasing}

% \begin{enumerate}
% \item An large-scale evaluation of state-of-the-art type error
%   localization techniques on over 4,500 student programs.
%   %
%   We collected a time-series of \ocaml programs from our students, and
%   use their subsequent \emph{fixes} to ill-typed programs as
%   \emph{oracles} for the location of the error.
%   %
%   This is the most extensive evaluation of the accuracy of type error
%   localization techniques that we are aware of.
%   % based not on expert \emph{opinion}, but
%   % on the eventual \emph{fix} implemented by a novice user.
%   %
%   % \ES{need to explain \emph{why} this is superior..}
% \item A machine learning approach to \emph{modeling} (novice) type
%   errors that can outperform the state-of-the-art by 15--30\% in
%   localizing errors.
%   %
%   Our classifiers can be trained on a relatively small amount of data --- we
%   used a single class of around 50 students --- and appear to
%   generalize to other instances of the same class.
%   %
%   \ES{and hopefully completely separate classes!}
%   %
%   In contrast to many approaches to \emph{fault localization} from the
%   software engineering community, our model does not use any linguistic
%   modeling techniques, \eg \emph{n-grams} over the token stream; rather,
%   it relies entirely on features of the abstract syntax tree and a
%   partial typing derivation.
%   %
%   \ES{if there's time, investigate adding Wes' n-gram model}
% \end{enumerate}

%%% Local Variables:
%%% mode: latex
%%% TeX-master: "main"
%%% End:
