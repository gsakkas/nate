\section{Learning to Blame}
\label{sec:learning}

\subsection{Syntax}
\label{sec:syntax}
\begin{figure}[t]
$$
\begin{array}{rrcl}
\emphbf{Expressions}
  & e & ::=    & x \spmid \efun{x}{e} \spmid \eapp{e}{e} \spmid \elet{x}{e}{e} \\
  &   & \spmid & n \spmid \eplus{e}{e}\\
  &   & \spmid & b \spmid \eif{e}{e}{e} \\
  &   & \spmid & \epair{e}{e} \spmid \epcase{e}{x}{x}{e} \\
  &   & \spmid & \enil \spmid \econs{e}{e} \spmid \ecase{e}{e}{x}{x}{e} \\[0.05in]

\emphbf{Integers}
  & n & ::= &  0, 1, -1, \ldots \\[0.05in]

\emphbf{Booleans}
  & b & ::= &  \etrue \spmid \efalse \\[0.05in]

\emphbf{Types}
  & t & ::= & \alpha \spmid \tbool \spmid \tint \spmid \tfun{t}{t} \spmid \tprod{t}{t} \spmid \tlist{t} \\[0.05in]
\end{array}
$$
\caption{Syntax of \lang}
\label{fig:syntax}
\end{figure}

\begin{figure}[t]
\centering
$$
\begin{array}{lcl}
  \V          & \defeq & \List{\R}\\

  \featuresym & : & \List{e \to \R} \\
  \labelsym   & : & e \times e \to \List{e} \\
  \extractsym & : & \List{e \to \R} \to e \times e \to \List{\V \times \Runit} \\
  \trainsym   & : & \List{\V \times \B} \to \Model \\
  \evalsym    & : & \Model \to \V \to \Runit \\
  \midrule
  \blamesym   & : & \Model \to e \to \List{e \times \Runit}
\end{array}
$$
\caption{A high-level API for converting program pairs to feature vectors and labels}
\label{fig:api}
\end{figure}

%
\autoref{fig:syntax} describes the syntax of \lang, a simple lambda
calculus with integers, booleans, pairs, and lists.

\subsection{Features}
\label{sec:features}
The first issue we must tackle is formulating our learning task in
machine learning terms.
%
We are given expressions $e$, but the learning algorithms expect to work
with \emph{feature vectors} --- vectors of real numbers, where each
column describes a particular feature of the input.
%
Thus, our first task is to convert expressions to feature vectors.

We choose to model a program as a \emph{set} of feature vectors, where
each element corresponds a sub-expression in the program. We group the
features into four categories.

\paragraph{Local syntactic features}
These features describe the syntactic category of each sub-expression
$e$.
%
In other words, for each production of $e$ in \autoref{fig:syntax} we
introduce a feature that is enabled (set to $1$) \emph{iff} the
sub-expression was built with that production.

\paragraph{Contextual syntactic features}
These are like local syntactic features, but lifted to describe the
parent and children of the current sub-expression.
%
If a particular $e$ does not have children (\eg a variable $x$) or a
parent (\ie the root expression), we leave the corresponding features
disabled (set to $0$).
%
This gives us a notion of the \emph{context} in which an expression
occurs, similar to the \emph{n-grams} commonly found in linguistic
models.

Instead of just describing the immediate context, we could describe
whether a particular syntax element occurs in the neighboring
sub-expressions (or even a count of how many times it occurs).
%
Such fuzzier notions of context may enable increased precision in the
model, but they also introduce opportunities for \emph{overfitting} ---
where the model memorizes particular inputs rather than learning general
patterns.
%
We will investigate (\ES{maybe..}) the impact of these alternatives
in \autoref{sec:evaluation}.

\paragraph{Expression size}
We also add a feature representing the \emph{size} of each expression,
\ie how many sub-expressions does it contain?
%
This allows the model to learn that, \eg, expressions closer to the
leaves are more likely to be blamed than expressions closer to the root.

\paragraph{Typing features}


\subsection{Labels}
\label{sec:labels}

\subsection{Models}
\label{sec:models}





%%% Local Variables:
%%% mode: latex
%%% TeX-master: "main"
%%% End:
