\section{Learning to Blame}
\label{sec:learning}
Next, we describe our approach to localizing type errors, in the context
of \lang (\autoref{fig:syntax}), a simple lambda calculus with integers,
booleans, pairs, and lists.
%
\begin{figure}[t]
$$
\begin{array}{rrcl}
\emphbf{Expressions}
  & e & ::=    & x \spmid \efun{x}{e} \spmid \eapp{e}{e} \spmid \elet{x}{e}{e} \\
  &   & \spmid & n \spmid \eplus{e}{e}\\
  &   & \spmid & b \spmid \eif{e}{e}{e} \\
  &   & \spmid & \epair{e}{e} \spmid \epcase{e}{x}{x}{e} \\
  &   & \spmid & \enil \spmid \econs{e}{e} \spmid \ecase{e}{e}{x}{x}{e} \\[0.05in]

\emphbf{Integers}
  & n & ::= &  0, 1, -1, \ldots \\[0.05in]

\emphbf{Booleans}
  & b & ::= &  \etrue \spmid \efalse \\[0.05in]

\emphbf{Types}
  & t & ::= & \alpha \spmid \tbool \spmid \tint \spmid \tfun{t}{t} \spmid \tprod{t}{t} \spmid \tlist{t} \\[0.05in]
\end{array}
$$
\caption{Syntax of \lang}
\label{fig:syntax}
\end{figure}

\begin{figure}[t]
\centering
$$
\begin{array}{lcl}
  \V          & \defeq & \List{\R}\\

  \featuresym & : & \List{e \to \R} \\
  \labelsym   & : & e \times e \to \List{e} \\
  \extractsym & : & \List{e \to \R} \to e \times e \to \List{\V \times \Runit} \\
  \trainsym   & : & \List{\V \times \B} \to \Model \\
  \evalsym    & : & \Model \to \V \to \Runit \\
  \midrule
  \blamesym   & : & \Model \to e \to \List{e \times \Runit}
\end{array}
$$
\caption{A high-level API for converting program pairs to feature vectors and labels}
\label{fig:api}
\end{figure}

%
First, we define the inputs to the model, a set of \emph{features}
(\autoref{sec:features}) that we will use to describe programs.
%
Second, we define the expected outputs of the model, a set of
\emph{labels} (\autoref{sec:labels}) that we will use to assign blame.
%
Third, we describe the actual \emph{models} (\autoref{sec:models}) that we
will evaluate.

\subsection{Features}
\label{sec:features}

\begin{figure}[ht]
\begin{minipage}{0.6\linewidth}
\begin{lstlisting}
  let rec sumList xs =
    match xs with
    | []     -> []
    | hd::tl -> hd + sumList tl
\end{lstlisting}
\end{minipage}
\begin{minipage}{0.3\linewidth}
\begin{lstlisting}[numbers=left]
[]
hd + sumList tl
sumList tl
tl
\end{lstlisting}
\end{minipage}
\caption{An ill-typed program (left) and a selection of its
  sub-expressions (right).}
\label{fig:sumList}
\end{figure}
\begin{table}[ht]
\begin{tabular}{lrrrrrrr}
\toprule
Expression
  & \IsNil & \IsCaseListP & \CountVarP & \ExprSize
  & \HasTypeIntCOne & \HasTypeList & \InSlice \\
\midrule
\lstinline![]!
  & 1 & 1 & 3 & 1 & 0 & 1 & 1 \\
\lstinline!hd + sumList tl!
  & 0 & 1 & 3 & 5 & 1 & 0 & 1 \\
\lstinline!sumList tl!
  & 0 & 0 & 3 & 3 & 0 & 1 & 1 \\
\lstinline!tl!
  & 0 & 0 & 2 & 1 & 0 & 1 & 0 \\
\bottomrule
\end{tabular}
\caption{A selection of the features we would extract from the
  \lstinline!sumList! program in \autoref{fig:sumList}. A feature is
  considered \emph{enabled} if it has a non-zero value, and
  \emph{disabled} otherwise. A ``-P'' suffix indicates that the feature
  describes the parent of the current expression, a ``-C$n$'' suffix
  indicates that the feature describes the $n$-th (left-to-right) child
  of the current expression.  Note that, since we rely on a partial
  typing derivation, we are subject to the well-known traversal bias and
  label the expression \lstinline!sumList tl! as having type
  $\tlist{\cdot}$. The model will have to learn to correct for this
  bias.}
\label{tab:sumList}
\end{table}

The first issue we must tackle is formulating our learning task in
machine learning terms.
%
We are given programs over \lang, but the learning algorithms expect to work
with \emph{feature vectors} --- vectors of real numbers, where each
column describes a particular aspect of the input.
%
Thus, our first task is to convert programs to feature vectors.

We choose to model a program as a \emph{set} of feature vectors, where
each element corresponds an expression in the program.
%
Thus, given the \lstinline!sumList! program in \autoref{fig:sumList} we
would first split it into its constituent sub-expressions and then
transform each sub-expression into a single feature vector.
%
We group the features into five categories, using \autoref{tab:sumList}
as a running example of the feature extraction process.

\paragraph{Local syntactic features}
These features describe the syntactic category of each expression $e$.
%
In other words, for each production of $e$ in \autoref{fig:syntax} we
introduce a feature that is enabled (set to $1$) if the expression was
built with that production, and disabled (set to $0$) otherwise.
%
For example, the \IsNil feature in \autoref{tab:sumList} describes
whether the current expression is the empty list $\enil$.

We distinguish between matching on a list vs.\ on a pair, as this
affects the typing derivation.
%
We also assume that all pattern matches are well-formed --- \ie all
patterns must match on the same type.
%
Ill-formed match expressions would lead to a type error; however, they
are already effectively localized to the match expression itself.
%
This is not a \emph{fundamental} limitation though, one could easily add
features that specify whether a match \emph{contains} a particular
pattern, and thus have a match expression that enables multiple
features. \ES{This is actually sort of what we do, we just ignore
  programs that have patterns of different types..}

\paragraph{Contextual syntactic features}
These are like local syntactic features, but lifted to describe the
parent and children of the current expression.
%
For example, the \IsCaseListP feature in \autoref{tab:sumList} describes
whether current expression's \emph{parent} matches on a list.
%
If a particular $e$ does not have children (\eg a variable $x$) or a
parent (\ie the root expression), we leave the corresponding features
disabled.
%
This gives us a notion of the \emph{context} in which an expression
occurs, similar to the \emph{n-grams} commonly found in linguistic
models.

Instead of just describing the immediate context, we could describe
whether a particular syntax element occurs in the neighboring
expressions (or even a count of how many times it occurs).
%
For example, the \CountVarP feature in \autoref{tab:sumList} describes
how many variables are contained in the expression \emph{rooted} at the
current expression's parent.
%
Such fuzzier notions of context may enable increased precision in the
model, but they also introduce opportunities for \emph{overfitting} ---
where the model memorizes particular inputs rather than learning general
patterns.
%
We will investigate (\ES{maybe..}) the impact of these alternatives
in \autoref{sec:evaluation}.

\paragraph{Expression size}
We also add a feature representing the \emph{size} of each expression,
\ie how many sub-expressions does it contain?
%
For example, the \ExprSize feature in \autoref{tab:sumList} is set to 3
for the expression \lstinline!sumList tl! as it contains 3 expressions:
the two variables and the application itself.
%
This allows the model to learn that, \eg, expressions closer to the
leaves are more likely to be blamed than expressions closer to the root.

\paragraph{Typing features}
A natural way of summarizing the context in which an expression occurs
is with \emph{types}.
%
Of course, the programs we are given are \emph{untypeable}, but we can
still extract a \emph{partial} typing derivation from the type checker
and use it to provide more information to the model.

A difficulty that arises here is that, due to the parametric type
constructors $\tfun{\cdot}{\cdot}$, $\tprod{\cdot}{\cdot}$, and
$\tlist{\cdot}$, there is an \emph{infinite} set of possible types ---
but we must have a \emph{finite} set of features.
%
Thus, we abstract the type of an expression to the set of type
constructors it \emph{mentions}, and add features for each type
constructor that describe whether a given type mentions the type
constructor.
%
For example, the type $\tint$ would only enable the $\tint$ feature,
while the type $\tfun{\tint}{\tbool}$ would enable the
$\tfun{\cdot}{\cdot}$, $\tint$, and $\tbool$ features.

We add these features for parent and child expressions to summarize the
context, but also for the current expression, as the type of an
expression is not always clear \emph{syntactically}.
%
For example, the expressions \lstinline!tl! and \lstinline!sumList tl!
in \autoref{tab:sumList} both enable the feature \HasTypeList, as they
are both inferred to have a type that mentions the $\tlist{\cdot}$
constructor.

Note that our use of typing features in an \emph{ill-typed} program
subjects us to the infamous traversal
bias~\cite{McAdam1998-ub,Yang2000-yr} --- in particular with the
\lstinline!sumList tl! expression, which might alternatively be assigned
the type $\tint$.
%
We rely on the model to counteract this bias.

\paragraph{Type error slice}
Finally, we would like the model to be able to distinguish between
changes that could fix the error, and changes that
\emph{cannot possibly} fix the error.
%
Thus, we compute a minimal type error \emph{slice} for the program
(\ie the set of expressions that contribute to the error), and add a
feature that is enabled \emph{iff} an expression is part of the slice.
%
The \InSlice feature in \autoref{tab:sumList} indicates whether an
expression is part of such a minimal slice, and is enabled for all of
the sampled expressions except for \lstinline!tl!, which does not affect
the type error.
%
If the program contains multiple type errors, we will compute
a minimal slice for each error.

\subsection{Labels}
\label{sec:labels}
We define the outputs of the model to be a simple ``good'' or ``bad''
label, where ``good'' means the expression \emph{should not} change and
``bad'' means the expression \emph{should} change.
%
This allows us to predict whether any individual expression should
change, but we would actually like to predict the \emph{most likely}
expressions to change.
%
Many learning algorithms produce not only a prediction, but also a
metric that can be interpreted as a \emph{confidence} in the prediction.
%
Thus, we add a post-processing step where we \emph{rank} the expressions
by the model's confidence that they will change, and select only the top
$k$ (in practice $k=3$) to present to the user.

\subsection{Learning Algorithms}
\label{sec:models}
\ES{TODO: mention that this is a SUPERVISED, CLASSIFICATION problem}
%
There are a large number of learning algorithms to choose from, existing
on a spectrum that balances expressiveness with ease of training (and of
interpreting the learned model).
%
In this section we present the learning algorithms we will investigate,
beginning with the simplest.
%
For each algorithm, we describe \ES{TODO...}

\paragraph{Logistic regression}
The simplest model we investigate is logistic regression, which learns a
linear function of the features.
%
The goal is to learn a set of weights $W$ and biases $b$ such that the
function
%
$$
y = Wx + b
$$
%
effectively maps feature vectors $x$ to labels $y$.

\ES{TODO: talk about logistic regression and the loss function?}

Linear models like logistic regression are popular for their ease of
training and of interpreting the resulting model; however, they can be
limited in their applicability.
%
As the name implies, linear models may perform poorly when asked to
learn an inherently \emph{nonlinear} function.
%
This limitation can be mitigated to some extent by adding more features
to the model (\eg combinations of the existing features \ES{CITE?}), but
the fundamental issue remains.

\paragraph{Decision trees}
\ES{TODO: Maybe Huma?}

\paragraph{Random forests}
\ES{TODO: Maybe Huma?}

\paragraph{Neural networks}
The last (and most complex) model we discuss is Artificial Neural
Networks (ANN).

The fundamental building block of an ANN is the linear model, the
nonlinearity arises from the \emph{activation functions} that govern
whether an individual linear model propagates its signal to the next layer.





%%% Local Variables:
%%% mode: latex
%%% TeX-master: "main"
%%% End:
