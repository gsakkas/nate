\mysection{Conclusion}
\label{sec:conclusion}

We have presented \toolname, which
combines modern statistical methods
with domain-specific feature engineering
to open the door to a new data-driven
path to precise error localization,
that significantly outperforms the
state of the art on a new benchmark
suite comprising 4500 student programs.
%
%%% RJNUKEDME   \toolname
%%% RJNUKEDME   translates pairs of
%%% RJNUKEDME   blame-labeled ill-typed
%%% RJNUKEDME   programs into bags-of-abstracted-terms,
%%% RJNUKEDME   and then trains a classifier
%%% RJNUKEDME   to predict whether a feature vector
%%% RJNUKEDME   should get a blame label.
%%% RJNUKEDME   % that
%%% RJNUKEDME   % predicts the value of the
%%% RJNUKEDME   % whether a feature vector gets
%%% RJNUKEDME   % a blame label.
%%% RJNUKEDME   %
%%% RJNUKEDME   Given a new ill-typed program,
%%% RJNUKEDME   we extract its bag-of-abstracted-terms
%%% RJNUKEDME   to compute the blame label for each
%%% RJNUKEDME   term and rank the predictions based
%%% RJNUKEDME   on the model's confidence, to select
%%% RJNUKEDME   the top 1--3 blame assignments to
%%% RJNUKEDME   present to the user.
%%% RJNUKEDME
%%% RJNUKEDME   We have evaluated our approach with
%%% RJNUKEDME   five different classifiers on a suite
%%% RJNUKEDME   of over 4,500 student programs.
%%% RJNUKEDME   %
%%% RJNUKEDME   We found that while machine learning
%%% RJNUKEDME   over syntactic features of each term in isolation
%%% RJNUKEDME   performs worse than existing
%%% RJNUKEDME   purely constraint based approaches, % (\eg \ocaml, \sherrloc),
%%% RJNUKEDME   augmenting the data with a single feature corresponding to
%%% RJNUKEDME   the type error slice brings our
%%% RJNUKEDME   classifiers up to par with the state of the art,
%%% RJNUKEDME   and further augmenting the data with
%%% RJNUKEDME   features of an expression's parent and children
%%% RJNUKEDME   allows our classifiers to outperform
%%% RJNUKEDME   the state of the art by 18 percentage points.

% We find that while machine
% learning on its own performs slightly worse
% than existing purely constraint based
% approaches (\eg \ocaml, \sherrloc), once
% we augment the data with feature abstraction
% corresponding to the \emph{type error slice}~\cite{Tip2001-qp},
% the resulting models models outperform
% the state of the art by up to 18
% percentage points.


As with other forms of machine learning,
a key concern is that of \emph{data-set bias}: are
\toolname's models specific
to our data set, would they fail on
\emph{other} programs?
%
We address this concern in two ways.
%
First, we partition the data by year,
and show that models learned from one
year generalize to, \ie perform nearly
as well on, the programs from the other
year.
%
Second, we argue that in our setting
this bias is a \emph{feature} (and not
a bug): it allows \toolname to \emph{adapt}
to the kinds of errors that programmers
(specifically novices, who are in greatest
need of precise feedback) actually make,
rather than hardwiring the biases of
experts who % , by dint of their
% training and experience,
may suffer from blind spots. % with regards to such problems.
%
In this regard, we are particularly pleased
that our classifiers can be trained on a
modest amount of data, \ie a single course's
worth, and envision a future where each course
comes equipped with a model of its students' errors.



%%% Local Variables:
%%% mode: latex
%%% TeX-master: "main"
%%% End:
