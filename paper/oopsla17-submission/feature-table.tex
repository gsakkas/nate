
\begin{table}[ht]
\caption{Example Feature Vectors}\label{tab:sumList}
\begin{tabular}{lrrrrrr}
\toprule
\textbf{Expression}
  & \IsNil & \IsCaseListP & \ExprSize
  & \HasTypeIntCOne & \HasTypeList & \InSlice \\
\midrule
|[]|
  & 1 & 1 & 1 & 0 & 1 & 1 \\
|hd + sumList tl|
  & 0 & 1 & 5 & 1 & 0 & 1 \\
|sumList tl|
  & 0 & 0 & 3 & 0 & 1 & 1 \\
|tl|
  & 0 & 0 & 1 & 0 & 1 & 0 \\
\bottomrule
\end{tabular}
\bigskip
\caption*{A selection of the features we would extract from the
\lstinline!sumList! program in \autoref{fig:sumList}. A feature is
considered \emph{enabled} if it has a non-zero value, and
\emph{disabled} otherwise. A ``-P'' suffix indicates that the feature
describes the parent of the current expression, a ``-C$n$'' suffix
indicates that the feature describes the $n$-th (left-to-right) child of
the current expression.  Note that, since we rely on a partial typing
derivation, we are subject to the well-known traversal bias and label
the expression \lstinline!sumList tl! as having type
$\tlist{\cdot}$. The model will have to learn to correct for this bias.}
\end{table}
