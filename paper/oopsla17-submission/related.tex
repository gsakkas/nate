\section{Related Work}
\label{sec:related-work}

In this section we describe two relevant aspects of related work:
software engineering approaches to fault localization, and 
programming languages approaches to pinpointing type errors. 

Given a defect, fault localization is the task of identifying
``suspicious'' program elements (e.g., lines, statements) that are likely
implicated in the defect (i.e., that should be changed to fix the defect). 
Perhaps the best-known is Tarantula, which uses a simple mathematical
formula based on measured information from dynamic normal and buggy
runs~\cite{FIXME}. Other similar approaches, including those of
Jacquard~\cite{FIXME} and Ochiai~\cite{FIXME} consider alternate features
of information or refined formulae and generally obtain more precise
results; see Wong for a survey~\cite{FIXME}. While some researchers have
approached such fault localization with an eye toward optimality (e.g.,
Harman et al. determine optimal coefficients~\cite{FIXME}), in general
such fault localization approaches are limited by their reliance on either
running tests or including relevant features. For example, Tarantula-based 
techniques require a normal and a buggy run of the program. By contrast,
we consider incomplete programs with type errors that may not be executed
in any standard sense. Similarly, the features available influence the
classes of defects that can be localized. A fault localization scheme based
purely on control flow features lines will have difficulty with cross-site
scripting or SQL code injection attacks, which follow the same control flow
path on normal and buggy runs (differing only in the user-supplied data).
By contrast, we FIXME-ERIC.

Languages with static type systems and type inference produce type errors
that novices often perceive as difficult to interpret~\cite{FIXME}.  For
example, approaches based on Hindley-Milner or other constraint systems 
typically have the issue that when the constraint $A=B$ is violated, 
the system cannot know whether the user should intend to fix $A$ or $B$ and
must thus report the discrepancy in a generic manner. As a result, a number
of approaches have been proposed to either give clearer error messages
(e.g.,~\cite{FIXME}) or more precisely localize the type error
(e.g.,~\cite{FIXME}). The technique most related to our work is FIXME-ERIC. 
It has been shown to be quite good at FIXME-ERIC. However, we choose to 
focus on FIXME-ERIC, and our approach contains FIXME-ERIC, which is not
present in previous work. 

%%% Local Variables:
%%% mode: latex
%%% TeX-master: "main"
%%% End:
