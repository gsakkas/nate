\subsection{Blame Accuracy}
\label{sec:quantitative}

In this experiment we compare the accuracy of our predictions to the
state of the art in type error localization.

\paragraph{Baseline}
We provide two baselines for the comparison: a random choice of location
from the minimized type error slice, and the standard \ocaml compiler.

\paragraph{State of the Art}
\mycroft~\citep{Loncaric2016-uk} localizes type errors by searching for
a minimal subset of typing constraints that can be removed, such that
the resulting system is satisfiable.
%
When multiple such subsets exist it can enumerate them, though it has no
notion of which subsets are \emph{more likely} to be correct, and thus
the order is arbitrary.
%
\sherrloc~\citep{Zhang2014-lv} localizes errors by searching the typing
constraint graph for constraints that participate in many unsatisfiable
paths and few satisfiable paths.
%
It can also enumerate multiple predictions, in descending order of
likelihood.

Comparing source locations from multiple tools with their own parsers is
not trivial.
%
Our experimental design gives the state of the art tools the ``benefit
of the doubt'' in two ways.
% To ensure a fair comparison when evaluating \mycroft and
% \sherrloc, 
First, when evaluating \mycroft and \sherrloc, we did not consider
programs where they predicted locations that our oracle could not match
with a program expression: 6--8\% of programs for \mycroft and 3--4\%
for \sherrloc.
%
Second, we similarly ignored programs where \mycroft or \sherrloc timed
out (after one minute) or where they encountered an unsupported language
feature: another 5\% for \mycroft and 12--13\% for \sherrloc.
%


\paragraph{Our Classifiers}
We evaluate five classifiers, each trained on the full set of features.
% features: 44 local syntactic features, 176 contextual syntactic
% features, 55 typing features, and a single expression size feature.
% %
% \ES{should explain the make-up of these groups}
%
% We preemptively discard expressions that are not part of the minimal
% type error slice --- we will explain the rationale for this in
% \autoref{sec:feature-utility} --- and thus the final feature count is
% 276.
%
Our classifiers are:
%
\begin{description}
\item[\linear] A logistic regression trained with a learning rate
  $\eta = 0.001$, an $L_2$ regularization rate $\lambda = 0.001$, and a
  mini-batch size of 200.
\item[\dectree] A decision tree trained with the CART algorithm
  \citep{Breiman1984-qy} and an impurity threshold of $10^{-7}$ (used to
  avoid overfitting via early stopping).
\item[\forest] A random forest \citep{Breiman2001-wo} of 30
  estimators, trained with an impurity threshold of $10^{-7}$.
\item[\hiddenT and \hiddenFH] Two multi-layer perceptron neural
  networks, both trained with $\eta = 0.001$, $\lambda = 0.001$, and a
  mini-batch size of 200.
  %
  The first MLP contains a single hidden layer of 10 neurons, and the
  second contains a hidden layer of 500 neurons.
  %
  This gives us a measure of the complexity of the MLP's model, \ie
  if the model requires many compound features, one would expect \hiddenFH
  to outperform \hiddenT.
  % This allows us to investigate how well the MLP can \emph{compress} its
  % model (cf.~\cite{FIXME}).
  %
  The neurons use rectified linear units (ReLU) as their activation
  function, a common practice in modern neural networks.
\end{description}
%
All classifiers were trained for 20 epochs on one dataset
--- \ie they were shown each program 20 times ---
before being evaluated on the other.
%
The logistic regression and MLPs were trained with the \textsc{Adam}
optimizer \citep{Kingma2014-ng}, a variant of stochastic gradient
descent that has been found to converge faster.


% colors from http://colorbrewer2.org/?type=sequential&scheme=Blues&n=3
\definecolor{blue1}{HTML}{DEEBF7}
\definecolor{blue2}{HTML}{9ECAE1}
\definecolor{blue3}{HTML}{3182BD}
\definecolor{green1}{HTML}{E5F5E0}
\definecolor{green2}{HTML}{A1D99B}
\definecolor{green3}{HTML}{31A354}

% \begin{figure}[ht]
% \centering
% \begin{tikzpicture}
% \begin{axis}[
%   % ybar stacked,
%   width=12cm,
%   height=8cm,
%   title={Impact of Feature Set on Accuracy},
%   ylabel={Accuracy},
%   %ymin=0.2,
%   ymax=1,
%   yticklabel={\pgfmathparse{\tick*100}\pgfmathprintnumber{\pgfmathresult}\,\%},
%   ytick style={draw=none},
%   ymajorgrids = true,
%   symbolic x coords={op+type+size, op+context+type+size, op+context-has+type+size, op+context-count+type+size},
%   % enlarge x limits=0.25,
%   xtick=data,
%   xtick style={draw=none},
%   xticklabels={Type, Context-Is, Context-Has, Context-Count},
%   x tick label style={rotate=45},
%   reverse legend,
%   transpose legend,
%   legend style={legend pos = outer north east, legend columns=4},
% ]
% % \addplot[draw=black, fill=blue1] table[x=tool, y=top-1] {\HiddenBench};
% % \addplot[draw=black, fill=blue2] table[x=tool, y expr=\thisrow{top-2} - \thisrow{top-1}] {\HiddenBench};
% % \addplot[draw=black, fill=blue3] table[x=tool, y expr=\thisrow{top-3} - \thisrow{top-2}] {\HiddenBench};

% \addplot[mark options={fill=blue1, scale=1.5}, mark=square*]
%   table[x=features, y=top-1] {\FeatureHiddenFHBench};
% \addplot[mark options={fill=blue2, scale=1.5}, mark=square*]
%   table[x=features, y=top-2] {\FeatureHiddenFHBench};
% \addplot[mark options={fill=blue3, scale=1.5}, mark=square*]
%   table[x=features, y=top-3] {\FeatureHiddenFHBench};
% \addlegendentry{Top-1}
% \addlegendentry{Top-2}
% \addlegendentry{Top-3}
% \addlegendimage{empty legend}
% \addlegendentry{\hiddenFH}

% \addplot[mark options={fill=blue1, scale=1.5}, mark=*]
%   table[x=features, y=top-1] {\FeatureLinearBench};
% \addplot[mark options={fill=blue2, scale=1.5}, mark=*]
%   table[x=features, y=top-2] {\FeatureLinearBench};
% \addplot[mark options={fill=blue3, scale=1.5}, mark=*]
%   table[x=features, y=top-3] {\FeatureLinearBench};
% \addlegendentry{Top-1}
% \addlegendentry{Top-2}
% \addlegendentry{Top-3}
% \addlegendimage{empty legend}
% \addlegendentry{\linear}

% \end{axis}
% \end{tikzpicture}
% \caption{reuslts!}
% \label{fig:results}
% \end{figure}

% \begin{figure}[ht]
% \centering
% \begin{tikzpicture}
% \begin{axis}[
%   ybar stacked,
%   width=12cm,
%   height=8cm,
%   title={Impact of Hidden Layer Size on Accuracy},
%   ylabel={Accuracy},
%   bar width=20pt,
%   %ymin=0.2,
%   ymax=1,
%   yticklabel={\pgfmathparse{\tick*100}\pgfmathprintnumber{\pgfmathresult}\,\%},
%   ytick style={draw=none},
%   ymajorgrids = true,
%   symbolic x coords={op+type+size/hidden-10, op+type+size/hidden-25, op+type+size/hidden-50,
%                      op+type+size/hidden-100, op+type+size/hidden-250, op+type+size/hidden-500},
%   % enlarge x limits=0.25,
%   xtick=data,
%   xtick style={draw=none},
%   xticklabels={\hiddenT, \hiddenTF, \hiddenF, \hiddenH, \hiddenTHF, \hiddenFH},
%   x tick label style={rotate=45},
%   reverse legend,
%   legend style={legend pos = north west},
% ]
% \addplot[draw=black, fill=blue1] table[x=tool, y=top-1] {\HiddenBench};
% \addplot[draw=black, fill=blue2] table[x=tool, y expr=\thisrow{top-2} - \thisrow{top-1}] {\HiddenBench};
% \addplot[draw=black, fill=blue3] table[x=tool, y expr=\thisrow{top-3} - \thisrow{top-2}] {\HiddenBench};
% % \addplot[draw=black, fill=blue1] table[x=tool, y=top-1] {\HiddenBench};
% % \addplot[draw=black, fill=blue2] table[x=tool, y=top-2] {\HiddenBench};
% % \addplot[draw=black, fill=blue3] table[x=tool, y=top-3] {\HiddenBench};
% \legend{Top-1, Top-2, Top-3}
% \end{axis}
% \end{tikzpicture}
% \caption{reuslts!}
% \label{fig:results}
% \end{figure}

\begin{figure}[t]
\centering
\begin{tikzpicture}
\begin{axis}[
  ybar stacked,
  width=\linewidth,
  height=7cm,
  title={Accuracy of Type Error Localization Techniques},
  ylabel={Accuracy},
  bar width=0.5cm,
  ymin=0,
  ymax=1,
  ytick={0.0, 0.1, 0.2, 0.3, 0.4, 0.5, 0.6, 0.7, 0.8, 0.9, 1.0},
  yticklabel={\pgfmathparse{\tick*100}\pgfmathprintnumber{\pgfmathresult}\,\%},
  ytick style={draw=none},
  ymajorgrids = true,
  symbolic x coords={baseline, ocaml, mycroft, sherrloc,
                     op+context+type+size/linear,
                     op+context+type+size/decision-tree,
                     op+context+type+size/random-forest,
                     op+context+type+size/hidden-10,
                     op+context+type+size/hidden-500},
  enlarge x limits=0.07,
  xtick=data,
  xtick style={draw=none},
  xticklabels={\random, \ocaml, \mycroft, \sherrloc,
               \linear, \dectree, \forest, \hiddenT, \hiddenFH},
  %x tick label style={rotate=45, anchor=north east},
  x tick label style={font=\small},
  y tick label style={font=\small},
  reverse legend,
  transpose legend,
  legend style={legend pos = north west, legend columns=4, font=\small},
]

% ES: NOTE: ORDER OF PLOTS/LEGEND ENTRIES MATTERS

\addplot[draw=black, fill=green1, bar shift=.25cm] table[x=tool, y=top-1] {\FallBench};
\addlegendentry{Top-1}
\addplot[draw=black, fill=green2, bar shift=.25cm] table[x=tool, y expr=\thisrow{top-2} - \thisrow{top-1}] {\FallBench};
\addlegendentry{Top-2}
\addplot[draw=black, fill=green3, bar shift=.25cm] table[x=tool, y expr=\thisrow{top-3} - \thisrow{top-2}] {\FallBench};
\addlegendentry{Top-3}
\addlegendimage{empty legend}
\addlegendentry{\FALL}

\resetstackedplots

\addplot[draw=black, fill=blue1, bar shift=-.25cm] table[x=tool, y=top-1] {\SpringBench};
\addlegendentry{Top-1}
\addplot[draw=black, fill=blue2, bar shift=-.25cm] table[x=tool, y expr=\thisrow{top-2} - \thisrow{top-1}] {\SpringBench};
\addlegendentry{Top-2}
\addplot[draw=black, fill=blue3, bar shift=-.25cm] table[x=tool, y expr=\thisrow{top-3} - \thisrow{top-2}] {\SpringBench};
\addlegendentry{Top-3}
\addlegendimage{empty legend}
\addlegendentry{\SPRING}


%\legend{Top-1, Top-2, Top-3}
\end{axis}
\end{tikzpicture}
\caption{
  %
  Results of our comparison of type error localization
  techniques.
  %
  We evaluate all techniques separately on two cohorts of
  students from different instances of an undergraduate
  Programming Languages course.
  %
  Our classifiers were trained on one cohort and evaluated on the other.
  %
  All of our classifiers outperform the state-of-the-art techniques
  \mycroft and \sherrloc.%  by a 10--15\% margin in Top-1 accuracy (with
%   the exception of \linear which is only slightly better than \sherrloc).
%
}
\label{fig:accuracy-results}
\end{figure}


\paragraph{Results}
\autoref{fig:accuracy-results} shows the results of our experiment.
%
Localizing the type errors in our benchmarks amounted, on average, to
selecting one of 3 correct locations out of a slice of 10.
%
Our classifiers consistently outperform the competition, ranging from
58--62\% Top-1 accuracy (86--88\% Top-3) for the \linear classifier to
71--74\% Top-1 accuracy (91\% Top-3) for the \hiddenFH.
%
Our baseline of selecting at random achieves 30\% Top-1
accuracy (58\% Top-3), while \ocaml achieves a Top-1 accuracy of 45\%.
%
Interestingly, one only needs two \emph{random} guesses to outperform
\ocaml, with 47\% accuracy.
%
\sherrloc outperforms both baselines, and comes close to our \linear
classifier, with 56\% Top-1 accuracy (84--86\% Top 3), while \mycroft
actually underperforms \ocaml with 38--41\% Top-1 accuracy.
%
% Finally, we find that \emph{all} of our classifiers outperform \sherrloc,
% ranging from 58--62\% Top-1 accuracy (86--88\% Top-3) for the \linear
% classifier to 71--74\% Top-1 accuracy (91\% Top-3) for the \hiddenFH.

Surprisingly, there is little variation in accuracy between our
classifiers.
%
With the exception of the \linear model, they all achieve around 70\%
Top-1 accuracy and around 90\% Top-3 accuracy.
%
This suggests that the model they learn is relatively simple.
%
In particular, notice that although the \hiddenT has $50\times$ \emph{fewer}
hidden neurons than the \hiddenFH, it only loses around 2\% accuracy.
% In particular, notice that the \hiddenT only loses around 2\% accuracy
% compared to the \hiddenFH,
%
We also note that our classifiers consistently perform better when
trained on the \FALL programs and tested on the \SPRING programs than
vice versa.
% , they appear to generalize better from the \FALL data.
% FIXME: Why? What is your explanation for this? Is it just sizes of those
% datasets or something qualitative about the program pairs in them?
