\mysection{Overview}\label{sec:overview}

%%%%%%%%%%%%%%%%%%%%%%%%%%%%%%%%%%%%%%%%%%%%%%%%%%%%%%%%%%%%%%%%%%%%%%%

\begin{figure}[t!]
\begin{minipage}{0.45\linewidth}
\begin{ecode}
  let rec sumList xs =
    match xs with
    | []   -> (*@\hlTree{\hlSherrloc{[]}}@*)
    | h::t -> h (*@\hlTree{+}@*) (*@\hlFix{sumList t}@*)
\end{ecode}
\end{minipage}
\begin{minipage}{0.49\linewidth}
\begin{verbatim}
File "sumList.ml", line 4, characters 16-25:
  This expression has type 'a list
  but an expression was expected of type int
\end{verbatim}
\end{minipage}
\caption{(left) An ill-typed \ocaml program that should sum the elements of a
  list, with highlights indicating three possible blame assignments based on:
  %
  (1) the \hlFix{\ocaml} compiler;
  %
  (2) the \hlSherrloc{fix} made by the programmer; and
  %
  (3) \hlTree{minimizing} the number of edits required.
  % The \underline{underlined} expressions are equally valid
  % locations to blame. The expression blamed by the \ocaml compiler
  % is in \textbf{bold}.
  %
  % FIXME: This bolding is ambiguous, because ``let rec'', ``match'' and
  % ``with'' are also bolded (by \\ecode)! You need to find another way to
  % highlight what ocaml is yelling about.
  %
  (right) The error reported by \ocaml.}
\label{fig:sumList}
\end{figure}

%%%%%%%%%%%%%%%%%%%%%%%%%%%%%%%%%%%%%%%%%%%%%%%%%%%%%%%%%%%%%%%%%%%%%%%

Let us start with an overview of \toolname's
approach to localizing type errors by
collectively learning from the mistakes
programmers actually make.

\mypara{The Problem}
%
Consider the |sumList| program in
\autoref{fig:sumList}, written by
a student in an undergraduate
Programming Languages course.
%
The program is meant to add up the
integers in a list, but the student
has accidentally given the empty
list as the base case, rather than |0|.
%
The \ocaml compiler collects typing
constraints as it traverses the program,
and reports an error the moment it finds
an inconsistent constraint.
%
In this case it blames the recursive call
to |sumList|, complaining that |sumList|
returns a |list| where an |int| was
expected by the |+| operator.
%
This \emph{blame} assignment is inconsistent
with the programmer's intention and may
not be helpful debugging information to
a novice.

It may appear obvious to the reader that
|[]| is the correct expression to blame,
but how is a type checker to know that?
%
Indeed, % both the venerable
% \ocaml compiler as well as
recent techniques like
\sherrloc and \mycroft
\citep{Zhang2014-lv,Loncaric2016-uk,Pavlinovic2014-mr}
fail to distinguish between
the |[]| and |+| expressions
in \autoref{fig:sumList};
it would be equally valid
to blame \emph{either}
of them alone.
%
The |[]| on line 3 could be changed to |0|,
or the |+| on line 4 could be changed to
either |@| (list append) or |::|, all of
which would give type-correct programs.
%
Thus, these state-of-the-art techniques
are forced to either blame \emph{both}
locations, or choose one \emph{arbitrarily}.

\mypara{Error Localization via Supervised Classification}
%
Our approach is to view error localization as a
\emph{supervised classification}
problem~\citep{Kotsiantis2007-pj}.
%
A \emph{classification} problem entails learning
a function that maps \emph{inputs} to a discrete
set of output \emph{labels} (in contrast to %, say,
\emph{regression}, where the output is typically
a real number).
%
A \emph{supervised} learning problem is one where
we are given a \emph{training set} where the
inputs and labels are known, and the task is to
learn a function that accurately maps the inputs
to output labels and \emph{generalizes} to new,
yet-unseen inputs.
%
To realize the above approach for error localization
as a practical tool, we have to solve four sub-problems.
%
\begin{enumerate}
  \item How can we acquire a \emph{training set} of
        blame-labeled ill-typed programs?

  \item How can we \emph{represent} blame-labeled programs
        as numeric vectors amenable to machine learning?

  \item How can we find \emph{features} that yield predictive
        models?

  \item How can we use the models to give localized
        \emph{feedback} to the programmer?
\end{enumerate}

\mysubsection{Step 1: Acquiring a Blame-Labeled Training Set}

The first step is to gather a training
set of ill-typed programs, where each
erroneous sub-term is explicitly labeled.
%
Prior work has often enlisted
expert users to curate a set of
ill-typed programs and then
\emph{manually} determine the
correct fix~\citep[\eg][]{Lerner2007-dt,Loncaric2016-uk}.
%
This method is suitable for
\emph{evaluating} the quality
of a localization (or repair)
algorithm on a small number
(\eg 10s--100s) of programs.
%
However, in general it requires
a great deal of effort for the
expert to divine the original
programmer's intentions.
%
Consequently, is difficult to
scale the expert-labeling to
yield a dataset large enough
(\eg 1000s of programs) needed
to facilitate machine learning.
%
More importantly, this approach
fails to capture the \emph{frequency}
with which errors occur in practice.

\mypara{Solution: Interaction Traces}
%
We solve both the scale and
frequency problems by instead
extracting blame-labeled data sets
from \emph{interaction traces}.
%
Software development is an iterative process.
Programmers, perhaps after a lengthy (and
sometimes frustrating) back-and-forth with
the type checker, eventually end up fixing
their own programs.
%
Thus, we instrumented
the \ocaml compiler to record
this conversation, \ie record the sequence
of programs submitted by each programmer and
whether or not it was deemed type-correct.
%
For each ill-typed program in
a particular programmer's trace,
we find the \emph{first subsequent}
program in the trace that type checks
and declare it to be the fixed version.
%
From this pair of an ill-typed program
and its fix, we can extract a \emph{diff}
of the abstract syntax trees, and then assign
the blame labels to the \emph{smallest}
sub-tree in the diff.

% \ES{FIXME: this feels a bit redundant with CHALLENGE 1. the main thing i
%  get is that we use student fixes to extract AST-diffs, which is
%  already said in CHALLENGE 1}

\mypara{Example}
Suppose our student
fixed the |sumList| program in
\autoref{fig:sumList} by replacing
|[]| with |0|, the diff would
include only the |[]| expression.
%
Thus we would determine that the
|[]| expression (and \emph{not} the
|+| or the recursive call |sumList t|)
is to blame.

%%%%%%%%%%%%%%%%%%%%%%%%%%%%%%%%%%%%%%%%%%%%%%%%%%%%%%%%%%%%%%%%%%%%%%%%%%%%%%%%

\mysubsection{Step 2: Representing Programs as Vectors}

Next, we must find a way to translate
highly structured and variable sized
\emph{programs} into fixed size
$n$-dimensional numeric \emph{vectors}
that are needed for supervised
classification.
%
While the PL literature is full
of different program
representations, from raw
sequences of tokens, to
richly-structured
abstract syntax trees (AST) or
control-flow graphs, it is
unclear how to embed the
above into a vector space.
%
Furthermore, it is unclear whether
recent program representations that
are amenable to one learning task,
\eg code completion \citep{Devanbu:2012,Raychev:2014}
or decompilation \citep{Raychev2015-jg,Bielik2016-br}
are suitable for our problem of
assigning blame for type errors.

\mypara{Solution: Bags-of-Abstracted-Terms}
%
We present a new representation of programs
that draws inspiration from the theory of
abstract interpretation \citep{CousotCousot77}.
%
Our representation as parameterized by a
set of \emph{feature abstraction} functions,
(abbreviated to feature abstractions)
$f_1, \ldots, f_n$, that map terms to a
numeric value (or just $\{0, 1\}$ to
encode a boolean property).
%
Given a set of feature abstractions, we
can represent a single program's AST as
a \emph{bag-of-abstracted-terms} (BOAT)
by:
%
(1)~decomposing the AST (term) $t$ into
    a \emph{bag} of its constituent sub-trees
    (terms) $\{t_1,\ldots,t_m\}$; and then
%
(2)~representing each sub-term $t_i$
    with the $n$-dimensional
    vector $[f_1(t_i),\ldots, f_n(t_i)]$.
%
Our choice of working with ASTs is natural
as that is the representation over which the
type-checker operates.



\mypara{Modeling Contexts}
%
Each expression occurs in some surrounding
\emph{context}, and we would like the
classifier to be able make decisions based
on the context as well.
%
The context is particularly important
for our task as each expression
imposes typing constraints on its
neighbors.
%
For example, a |+| operator tells
the type checker that both \emph{children}
must have type |int| and that the \emph{parent}
must accept an |int|.
%
Similarly, if the student wrote
|h sumList t| \ie forgot the |+|,
we might wish to blame the application
rather than |h| because |h|
\emph{does not} have a function type.
%
The BOAT representation makes it
easy to incorporate contexts: we
simply \emph{concatenate} each
term's feature vector with the
\emph{contextual features} of
its parent and children.

%% \ES{FIXME: this feels a bit redundant with CHALLENGE 2. the main things
  %% i get here are that we use sets of feature vectors, and that we can
  %% concatenate vectors to add context. but this is all said in CHALLENGE
  %% 2 already.}

%% \RJ{confusing -- delete}
%% We might expect that the recursive
%% call |sumList t| is unlikely to be
%% at fault because |sumList| has a
%% function type --- the user may have
%% called the wrong function or
%% supplied the wrong argument, but the
%% application itself is probably correct.

\mysubsection{Step 3: Feature Discovery}

Next, we must find a \emph{good}
set of features, that is, a set
of features that yields predictive
models. Our BOAT representation
lets us iteratively solve this
problem by just starting with
a simple set of features, and
then repeatedly adding more
and more to capture important
aspects needed to improve precision.
%
Our set of feature abstractions
%includes those that capture the
captures the
\emph{syntax}, \emph{types}, and
\emph{context} of each expression.

\mypara{Syntax and Type Features}
%
We start by observing that
at the very least, the
classifier should be able
to distinguish between the
|[]| and |+| expressions
in \autoref{fig:sumList}
because they represent
different \emph{syntactic}
expression forms.
%
We model this by
introducing feature
abstractions of the form
is-nil, is-plus, \etc for
each of a fixed number of
primitive literal terms.
%
Modeling the syntactic class of an
expression gives the classifier a
basic notion of the relative
frequency of blame assignment
for the various program elements,
\ie perhaps |[]| is
\emph{empirically} more
likely to be blamed than |+|.
%
Similarly, we can model
the \emph{type} of each
sub-expression with features
of the form is-int, is-bool, \etc.
%
We will discuss handling
arbitrary, user-defined types
in~\autoref{sec:discussion}.

\mypara{Contextual Features: Error Slices}
%
Our contextual features include the
syntactic class of the neighboring
expressions and their inferred types
(when available).
%
However, we have found that
the most important contextual
signal is whether or not the
expression occurs in
a minimal type error slice
\citep{Tip2001-qp,Haack2003-vc}
which includes \emph{a minimal subset}
of all expressions that are
necessary for the error to manifest.
%
(That is, replacing any subterm
with |undefined| or |assert false|
would yield a well-typed program.)
%
We propose to use type error slices
to communicate to the classifier
which expressions could
\emph{potentially} be blamed --- a
change to an expression outside of
the minimal slice cannot possibly
fix the type error.
%
We empirically demonstrate that
the type error slice is so
important (\autoref{sec:feature-utility})
that it is actually beneficial to
automatically discard expressions
that are not part of the slice,
rather than letting the classifier
learn to do so.
%
Indeed, this domain-specific
insight is crucial for learning
classifiers that significantly
outperform the state-of-the-art.

\mypara{Example}
%
When \toolname is tasked with localizing
the error in the example program of \autoref{fig:sumList},
the |[]| and |+| sub-terms will each be given
their own feature vector, and we will ask the
classifier to predict for each \emph{independently}
whether it should be blamed.
%
\autoref{tab:sumList} lists some
of the sub-expressions of the example
from \autoref{fig:sumList}, and their
corresponding feature vectors.


\begin{table}[ht]
\caption{Example Feature Vectors}\label{tab:sumList}
\begin{tabular}{lrrrrrr}
\toprule
\textbf{Expression}
  & \IsNil & \IsCaseListP & \ExprSize
  & \HasTypeIntCOne & \HasTypeList & \InSlice \\
\midrule
|[]|
  & 1 & 1 & 1 & 0 & 1 & 1 \\
|hd + sumList tl|
  & 0 & 1 & 5 & 1 & 0 & 1 \\
|sumList tl|
  & 0 & 0 & 3 & 0 & 1 & 1 \\
|tl|
  & 0 & 0 & 1 & 0 & 1 & 0 \\
\bottomrule
\end{tabular}
\bigskip
\caption*{A selection of the features we would extract from the
\lstinline!sumList! program in \autoref{fig:sumList}. A feature is
considered \emph{enabled} if it has a non-zero value, and
\emph{disabled} otherwise. A ``-P'' suffix indicates that the feature
describes the parent of the current expression, a ``-C$n$'' suffix
indicates that the feature describes the $n$-th (left-to-right) child of
the current expression.  Note that, since we rely on a partial typing
derivation, we are subject to the well-known traversal bias and label
the expression \lstinline!sumList tl! as having type
$\tlist{\cdot}$. The model will have to learn to correct for this bias.}
\end{table}


% Our use of type error slices is
% conceptually related to the use
% of fault localization in other
% program analysis and transformation
% techniques (see \autoref{sec:related-work}).


\mysubsection{Step 4: Generating Feedback}

Finally, having trained a classifier
using the labeled data set, we need to use
it to help users localize type errors in
their programs.
%
The classifier tells us whether or not
a sub-term \emph{should be}
blamed (\ie has the blame label) but this
is not yet particularly suitable as
\emph{user feedback}.
%
A recent survey of developers by
\citet{Kochhar2016-oc} found that
developers are unlikely to examine
more than around five potentially
erroneous locations before falling
back to manual debugging.
%
Thus, we should limit our predictions
to a select few to be presented to
the user.

\mypara{Solution: Rank Locations by Confidence}
%
Fortunately, many machine learning
classifiers produce not only a predicted
label, but also a metric that can be
interpreted as the classifier's
\emph{confidence} in its prediction.
%
Thus, we \emph{rank} each expression
by the classifier's confidence that
it should be blamed, and present only
the top-$k$ predictions to the
user (in practice $k=3$).
%
This use of ranking to report the
results of a program analysis is
popular in other problem domains
\citep[see, \eg][]{Kremenek2003-ck};
we focus explicitly on the use of
data-driven machine learning
confidence as a ranking source.
%
In \autoref{sec:evaluation} we show
that \toolname's ranking approach
yields a high-precision localizer:
when the top-3 locations are considered,
at least one matches an actual student fix
more than 90\% of the time.


%%% HOOK TO EVALUATION: Evaluation


%%% Local Variables:
%%% mode: latex
%%% TeX-master: "main"
%%% End:
