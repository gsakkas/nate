\mysection{Qualitative Evaluation}
\label{sec:qualitative}

Next, we present a \emph{qualitative} evaluation
that compares the predictions made by our classifiers
with those of \sherrloc.
%
In particular, we demonstrate, with a series of example programs from
our student dataset, how our classifiers are able to use past student
mistakes to make more accurate predictions of future fixes.
%
We also take this opportunity to examine some of the specific features
our classifiers use to assign blame.
%
For each example, we provide
%
(1) the code;
%
(2) \hlSherrloc{\sherrloc's} prediction;
%
(3) our \hlTree{\dectree's} prediction; and
%
(4) an \emph{explanation} of why our classifier made its prediction, in
terms of the features used and their values.
%
We choose the \dectree classifier for this section as its model is more
easily interpreted than the MLP.\@
%
We also exclude the \ExprSize feature from the model used in this
section, as it makes the predictions harder to motivate, and as we saw
in \autoref{sec:feature-utility} it does not appear to contribute
significantly to the model's accuracy.

We explain the predictions by analyzing the paths induced in
the decision tree by the features of the input expressions.
%
Recall that each node in a decision tree contains a simple predicate of
the features, \eg ``is feature $v_j$ enabled?'', which determines
whether a sample will continue down the left or right subtree.
%
Thus, we can examine the predicates used and the values of the
corresponding features to explain \emph{why} our \dectree made its
prediction.
%
We will focus particularly on the enabled features, as they generally
provide more information than the disabled features.
%
Furthermore, each node is additionally labeled with the ratio of
``blamed'' vs ``not-blamed'' training expressions that passed through
it.
%
We can use this information to identify particularly important
decisions, \ie we consider a decision that changes the ratio to be more
interesting than a decision that does not.

%
% We will also attempt, for each program, to explain \emph{why} the
% Decision Tree made its prediction, by analyzing the paths induced
% by the programs.

\mysubsection{Failed Predictions}
\label{sec:failed-predictions}

We begin with a few programs where our classifier fails to
make the correct prediction.
%
For these programs we will additionally \hlFix{highlight} the
correct blame location.

\mypara{Constructing a List of Duplicates}
% data/sp14/0219.ml
Our first program is a simple recursive function |clone| that takes an
item |x| and a count |n|, and produces a list containing |n| copies of
|x|.
%
% \begin{ecode}
%   let rec clone x n =
%     let loop acc n =
%       if n <= 0 then
%         acc
%       else
%         (*@\colorbox{red!50}{clone}@*) ==([==_=x=_==] @ acc)== (n - 1) in
%     loop [] n
% \end{ecode}
\begin{ecode}
  let rec clone x n =
    let loop acc n =
      if n <= 0 then
        acc
      else
        (*@\hlFix{clone} \hlSherrloc{([\hlTree{x}] @\ acc)}@*) (n - 1) in
    loop [] n
\end{ecode}
% \ES{TODO: our 2nd prediction matches \sherrloc and \ocaml (occurs check), correct fix is to replace recursive call to clone with loop}
%
The student has defined a helper function |loop| with an accumulator
|acc|, likely meant to call itself tail-recursively.
%
Unfortunately, she has called the top-level function |clone| rather than
|loop| in the |else| branch, this induces a cyclic constraint |'a = 'a list|
for the |x| argument to |clone|.

Our classifier incorrectly predicts that the use of |x| in the recursive
call is the most likely source of the error.
%
Our second prediction coincides with \sherrloc (and \ocaml), blaming the
the first argument to |clone|.
%
This is also incorrect, but may be more helpful than our first
prediction --- if our student decides that she has certainly provided
the correct \emph{argument}, an alternative explanation is that
perhaps she has called the wrong \emph{function}.

We confess that both of these predictions are difficult to explain by
examining the induced paths.
%
In particular, both predictions only reference the expression's context,
which is surprising.
%
Much clearer is why we fail to blame the occurrence of |clone|, the two
enabled features on the path are:
%
(1) the parent is an application; and
%
(2) |clone| has a function type.
%
The model seems to have learned that programmers typically call the
correct function.

\mypara{Currying Considered Harmful?}
% data/sp14/1204.ml
Our next example is another ill-fated attempt at |clone|.
%
\begin{ecode}
  let rec clone x n =
    let rec loop (*@\hlFix{x n acc}@*) =
      if n < 0 then
        acc
      else
        (*@\hlSherrloc{loop} \hlTree{(x, (n - 1), (x :: acc))}@*) in
    loop (x, n, [])
\end{ecode}
%
The issue here is that \ocaml functions are \emph{curried} by default
--- \ie they take their arguments one at a time --- but our student has
called the inner |loop| with all three arguments in a tuple.
%
Many experienced functional programmers would choose to keep |loop|
curried and rewrite the calls, however our student decides instead to
\emph{uncurry} |loop|, making it take a tuple of arguments.
%
\sherrloc blames the recursive call to |loop| while our classifier
blames the tuple of arguments --- a reasonable suggestion, but not
the answer the student expected.

We fail to blame the definition of |loop| because it is defining a
function.
%
First, note that we represent |let f x y = e|\ \ as\ \ |let f = fun x -> fun y -> e|,
thus a change to the pattern |x| would be treated as a change to the outer
|fun| expression.
%
With this in mind, we can explain our failure to blame the definition of
|loop| (the outer |fun|) as follows:
%
(1) it has a function type;
%
(2) its child is a |fun|; and
%
(3) its parent is a |let|.
%
Thus it appears to the model that the outer |fun| is simply part of a
function definition, a common and innocuous phenomenon.

% \mypara{Composing Functions}
% % data/sp14/3041.ml
% Next, let us consider the |pipe| function that composes a list of
% functions, \ie given a list of functions |[f;g;h]|, |pipe| produces the
% function |fun x -> h (g (f x))|.
% %
% \begin{ecode}
%   let pipe fs =
%     let f a x y = ==y== __(a y)__ in
%     let base x = x in
%     List.fold_left f base fs
% \end{ecode}
% %
% The error in our student's code is that she has applied |y| rather than
% |x| to the result of |a y|.
% %
% \sherrloc correctly blames the first occurrence of |y|, while our
% classifier (incorrectly) blames the application |a y| (\ocaml blames
% the occurrence of |base| on line 4).
% %
% \ES{this has the same explanation as the 1st clone above}

\mysubsection{Correct Predictions}
\label{sec:correct-predictions}

Next, we present a few indicative programs where our first prediction is
correct, and all of the other tools' top three predictions are
incorrect.

\mypara{Extracting the Digits of an Integer}
% data/sp14/2176.ml
Consider first a simple recursive function |digitsOfInt| that extracts
the digits of an |int|.
%
\begin{ecode}
  let rec digitsOfInt n =
    if n <= 0 then
      []
    else
      [n mod 10] @ (*@\hlTree{[ \hlSherrloc{digitsOfInt (n / 10)} ]}@*)
\end{ecode}
%
Unfortunately, the student has decided to wrap the recursive call to
|digitsOfInt| with a list literal, even though |digitsOfInt| already
returns an |int list|.
%
Thus, the list literal is inferred to have type |int list list|, which
is incompatible with the |int list| on the left of the |@| (list append)
operator.
%
Both \sherrloc and the \ocaml compiler blame the recursive call for
returning a |int list| rather than |int|, but the recursive call is
correct!

As our \dectree correctly points out (with high confidence), the fault
lies with the list literal \emph{surrounding} the recursive call, remove
it and the type error disappears.
%
An examination of the path induced by the list literal reveals that our
\dectree is basing its decision on the fact that
%
(1) the expression is a list literal;
%
(2) the child expression is an application, whose return type mentions |int|; and
%
(3) the parent expression's type mentions |list|.
%
Interestingly, \dectree incorrectly predicts that the child application
should change as well, but it is less confident of this prediction and
ranks it below the correct blame assignment.

\mypara{Padding a list}
% data/sp14/0442.ml
Our next program, |padZero|, is given two |int list|s as input, and must
left-pad the shorter one with enough zeros that the two output lists
have equal length.

The student first defines a helper function |clone|, as seen above.
%
\begin{ecode}
  let rec clone x n =
    if n <= 0 then
      []
    else
      x :: clone x (n - 1)
\end{ecode}
%
Then she defines |padZero| with a simple branch that determines which
list is shorter, followed by a |clone| for the appropriate number of
zeros.
%
\lstset{firstnumber=last}
\begin{ecode}
  let padZero l1 l2 =
    let n = List.length l1 - List.length l2 in
    if n < 0 then
      (clone 0 ((-1) * n) @ l2, l2)
    else
      (l1, (*@\hlTree{\hlSherrloc{clone 0 n} :: l2}@*))
\end{ecode}
\lstset{firstnumber=1}
%
Alas, our student has accidentally used the |::| operator rather than
the |@| operator in the |else| branch.
%
\sherrloc and \ocaml correctly determine that she cannot cons the
|int list| returned by |clone| onto |l2|, which is another |int list|,
but they decide to blame the call to |clone|, while our
\dectree correctly blames the |::| constructor.

Examining the path induced by the |::| constructor, we can see that our
\dectree is influenced by the fact that:
%
(1) the expression is a data constructor (not specifically |::|);
%
(2) the parent expression is a tuple; and
%
(3) the leftmost child is an application.
%
We note that first fact appears to be particularly significant; an
examination of the training samples that reach that decision node
reveals that before observing the \textsc{Is-Constructor} feature, the
classifier is slightly in favor of predicting ``blame'', but afterwards
it is heavily in favor of predicting ``blame''.
%
Many of the following decisions change the balance back towards ``no
blame'' if the ``true'' path is taken, but the |::| constructor always
takes the ``false'' path.
%
It would appear that our \dectree has learned that data constructors are
particularly suspicious, and is looking for exceptions to this general
rule.
%
% \ES{FIXME: this is probably super confusing, checking with Huma to see
%   if we can get a nice graph to illustrate what's happening..}

Our \dectree correctly predicts that the recursive call blamed by
\sherrloc should not be blamed; a similar examination suggests that the
crucial observation is that the recursive call's parent is a data
constructor application.
%



% ES: decision tree gets this one wrong, may want to find something else
% \begin{ecode}
%   let rec sepConcat sep sl =
%     match sl with
%     | [] -> ""
%     | h::t ->
%         let f a x = a ^ (sep ^ x) in
%         List.fold_left f h t

%   let stringOfList f l = sepConcat "; " __[ "["; ___=List.map f l=___; "]" ]__
% \end{ecode}

% \ES{I don't much like this example anymore..}
% \mypara{Computing the Fixed Point of a Function}
% % data/sp14/1266.ml
% Finally, our students must write a |fixpoint| function that computes the
% fixed point of a given function |f|, starting from an initial value |b|.
% %
% As a hint, we first have them write a |wwhile| function that performs the
% functional equivalent of a \texttt{while}-loop, repeatedly passing a function's
% output back in until it receives a (boolean) signal to stop.
% %
% \begin{ecode}
%   let rec wwhile (f, b) =
%     match f b with
%     | (x, false) -> x
%     | (x, true)  -> wwhile (f, x)
% \end{ecode}
% \lstset{firstnumber=last}
% %
% The |fixpoint| function can then be written as a clever instantiation of
% the arguments to |wwhile|.
% %
% \begin{ecode}
%   let fixpoint (f, b) =
%     let g = __let bb = f b in ___=(bb, (bb = b))=_ in
%     wwhile (g, b)
% \end{ecode}
% \lstset{firstnumber=1}
% %
% Sadly, our student has forgotten that |g| should itself be a function.
% %
% As a result, she passes a pair of a \emph{pair} and a starting value to
% |wwhile|, rather than a pair of a \emph{function} and a starting value.
% %
% \sherrloc deduces that the call to |wwhile| is likely correct (\ocaml
% actually blames the use of |g|), but identifies the construction of the
% pair inside |g| as the most likely culprit, while our \dectree correctly
% identifies the \emph{definition} of |g| as the source of the error.
% %
% Our \dectree appears to base its prediction, with complete confidence,
% on the fact that:
% %
% (1) the parent expression is a |let|;
% %
% (2) the right child is a tuple; and finally
% %
% (3) the current expression is a |let|.
% %



%%% Local Variables:
%%% mode: latex
%%% TeX-master: "main"
%%% End:
