\section{Quantitative Evaluation}
\label{sec:evaluation}
\pgfplotstableset{col sep=comma}

% \pgfplotstableread{../../data/sp14/op+type+size/linear/results.csv}{\FeatureLinearBench}
% \pgfplotstablevertcat{\FeatureLinearBench}{../../data/sp14/op+context+type+size/linear/results.csv}
% \pgfplotstablevertcat{\FeatureLinearBench}{../../data/sp14/op+context-has+type+size/linear/results.csv}
% \pgfplotstablevertcat{\FeatureLinearBench}{../../data/sp14/op+context-count+type+size/linear/results.csv}
% \pgfplotstableread{../../data/sp14/op+type+size/hidden-10/results.csv}{\FeatureHiddenTBench}
% \pgfplotstablevertcat{\FeatureHiddenTBench}{../../data/sp14/op+context+type+size/hidden-10/results.csv}
% \pgfplotstablevertcat{\FeatureHiddenTBench}{../../data/sp14/op+context-has+type+size/hidden-10/results.csv}
% \pgfplotstablevertcat{\FeatureHiddenTBench}{../../data/sp14/op+context-count+type+size/hidden-10/results.csv}
% \pgfplotstableread{../../data/sp14/op+type+size/hidden-500/results.csv}{\FeatureHiddenFHBench}
% \pgfplotstablevertcat{\FeatureHiddenFHBench}{../../data/sp14/op+context+type+size/hidden-500/results.csv}
% \pgfplotstablevertcat{\FeatureHiddenFHBench}{../../data/sp14/op+context-has+type+size/hidden-500/results.csv}
% \pgfplotstablevertcat{\FeatureHiddenFHBench}{../../data/sp14/op+context-count+type+size/hidden-500/results.csv}

% \pgfplotstableread{../../data/sp14/op+type+size/hidden-10/results.csv}{\HiddenBench}
% \pgfplotstablevertcat{\HiddenBench}{../../data/sp14/op+type+size/hidden-25/results.csv}
% \pgfplotstablevertcat{\HiddenBench}{../../data/sp14/op+type+size/hidden-50/results.csv}
% \pgfplotstablevertcat{\HiddenBench}{../../data/sp14/op+type+size/hidden-100/results.csv}
% \pgfplotstablevertcat{\HiddenBench}{../../data/sp14/op+type+size/hidden-250/results.csv}
% \pgfplotstablevertcat{\HiddenBench}{../../data/sp14/op+type+size/hidden-500/results.csv}
% % \pgfplotstablevertcat{\HiddenBench}{../../data/sp14/op+context-count+type+size/hidden-10/results.csv}
% % \pgfplotstablevertcat{\HiddenBench}{../../data/sp14/op+context-count+type+size/hidden-25/results.csv}
% % \pgfplotstablevertcat{\HiddenBench}{../../data/sp14/op+context-count+type+size/hidden-50/results.csv}
% % \pgfplotstablevertcat{\HiddenBench}{../../data/sp14/op+context-count+type+size/hidden-100/results.csv}
% % \pgfplotstablevertcat{\HiddenBench}{../../data/sp14/op+context-count+type+size/hidden-250/results.csv}
% % \pgfplotstablevertcat{\HiddenBench}{../../data/sp14/op+context-count+type+size/hidden-500/results.csv}

\pgfplotstableread{../../data/sp14/baseline.csv}{\SpringBench}
\pgfplotstablevertcat{\SpringBench}{../../data/sp14/ocaml/results.csv}
\pgfplotstablevertcat{\SpringBench}{../../data/sp14/mycroft/results.csv}
\pgfplotstablevertcat{\SpringBench}{../../data/sp14/sherrloc/results.csv}
\pgfplotstablevertcat{\SpringBench}{../../data/sp14/op+context+type+size/linear/results.csv}
\pgfplotstablevertcat{\SpringBench}{../../data/sp14/op+context+type+size/decision-tree/results.csv}
\pgfplotstablevertcat{\SpringBench}{../../data/sp14/op+context+type+size/random-forest/results.csv}
\pgfplotstablevertcat{\SpringBench}{../../data/sp14/op+context+type+size/hidden-10/results.csv}
\pgfplotstablevertcat{\SpringBench}{../../data/sp14/op+context+type+size/hidden-500/results.csv}

\pgfplotstableread{../../data/fa15/baseline.csv}{\FallBench}
\pgfplotstablevertcat{\FallBench}{../../data/fa15/ocaml/results.csv}
\pgfplotstablevertcat{\FallBench}{../../data/fa15/mycroft/results.csv}
\pgfplotstablevertcat{\FallBench}{../../data/fa15/sherrloc/results.csv}
\pgfplotstablevertcat{\FallBench}{../../data/fa15/op+context+type+size/linear/results.csv}
\pgfplotstablevertcat{\FallBench}{../../data/fa15/op+context+type+size/decision-tree/results.csv}
\pgfplotstablevertcat{\FallBench}{../../data/fa15/op+context+type+size/random-forest/results.csv}
\pgfplotstablevertcat{\FallBench}{../../data/fa15/op+context+type+size/hidden-10/results.csv}
\pgfplotstablevertcat{\FallBench}{../../data/fa15/op+context+type+size/hidden-500/results.csv}


We have implemented our technique for localizing type errors for a
purely functional subset of \ocaml with polymorphic types and functions.
%
We seek to answer three questions in our evaluation:
%
\begin{description}
\item[1. Blame Accuracy (\autoref{sec:quantitative})]
  %
  How often do we \emph{correctly} assign blame for the type error?
  %
  % We compare our technique with a variety of off-the-shelf classifiers
  % and find that our top-ranked blame assignments have an accuracy of
  % 72\%, compared to a state-of-the-art 56\%.
  % For how many ill-typed programs can we accurately predict the source
  % of the error?
\item[2. Feature Utility (\autoref{sec:feature-utility})]
  %
  Which feature sets are \emph{most important} for accurate predictions?
  % How much do the features described in \autoref{sec:features}
  % contribute to our predictions?
\item[3. Explaining Predictions (\autoref{sec:qualitative})]
  %
  Can we \emph{explain} individual predictions (correct or not)?
\end{description}
%\ES{may want to swap (1) and (2)..}

\mysubsection{Methodology}
\label{sec:methodology}

We answer our questions on two sets of data gathered from the
undergraduate Programming Languages course at
\begin{anonsuppress}
UC San Diego (IRB \#140608).
\end{anonsuppress}
\begin{noanonsuppress}
AUTHOR's INSTITUTION.
\end{noanonsuppress}
%
We recorded each interaction with the \ocaml top-level system over the
course of the first three assignments, capturing
ill-typed programs and, crucially, their subsequent fixes.
%
The first dataset comes from the Spring 2014 class (\SPRING), with a
cohort of 46 students. The second comes from the Fall 2015 class
(\FALL), with a cohort of 56 students.
%
The extracted programs are relatively small, but they demonstrate a
range of functional programming idioms, \eg higher-order functions and
(polymorphic) algebraic data types.

\mypara{Feature Selection}
We extract a set of 277 features from each sub-expression in a student
program, including:
%
\begin{enumerate}
\item 44 local syntactic features. In addition to the syntax of \lang,
  we support the full range of arithmetic operators (integer and
  floating point), equality and comparison operators, character and
  string literals, and a user-defined |expr| type of simple arithmetic
  expressions. We discuss the challenge of supporting other
  user-defined types in \autoref{sec:discussion}.
\item 176 contextual syntactic features. For each sub-expression we
  additionally extract the local syntactic features of its parent and
  first, second, and third (left-to-right) children. If an expression
  does not have a parent or children, these features will simply be
  disabled. If an expression has more than three children, the
  classifiers will receive no information about the additional
  children.
\item 55 typing features. In addition to the types of \lang, we support
  |int|s, |float|s, |char|s, |string|s, and the user-defined |expr|
  mentioned above. These features are extracted for each sub-expression
  and for the contextual sub-expressions.
\item One feature denoting the size of each sub-expression.
\item One feature denoting whether each sub-expression is part of the
  minimal type error slice. We use this feature as a ``hard''
  constraint, sub-expressions that are not part of the minimal slice
  will be preemptively discarded. We justify this decision in
  \autoref{sec:feature-utility}.
\end{enumerate}

\mypara{Blame Oracle}
Recall from \autoref{sec:labels} that we automatically extract a blame
oracle for each ill-typed program from the (AST) diff between it and the
student's eventual fix.
%
A disadvantage of using diffs in this manner is that students may have
made many, potentially unrelated, changes between compilations; at some
point the ``fix'' becomes a ``rewrite''.
%
We do not wish to consider the ``rewrites'' in our evaluation, so we
discard outliers where the fraction of expressions that have changed is
more than one standard deviation above the mean, establishing a diff
threshold of 44\%.
%
This accounts for roughly 14\% of each dataset, leaving us with
2,425 program pairs for \SPRING and 2,325 pairs for \FALL.

% we discard any program pairs where more than 40\%
% of the sub-expressions have changed.
% %
% We picked 40\% as an estimate of the inflection point where we could
% still retain the large majority of program pairs.
% % FIXME: Can you say that this dataset curation is similar to any other
% % datasets (e.g., the washington one)? Anything you could cite and discuss
% % here would take some of the pressure off.


\mypara{Accuracy Metric}
All of the tools we compare (with the exception of the standard \ocaml
compiler) can produce a list of potential error locations.
%
However, in a study of fault localization techniques,
\citet{Kochhar2016-oc} show that most developers will not consider more
than around five potential error locations before falling back to manual
debugging.
%
Type errors are relatively simple in comparison to general fault
localization, thus we limit our evaluation to the top three predictions
of each tool.
%
We evaluate each tool on whether a changed expression occurred in its
top one, top two, or top three predictions.

\subsection{Blame Accuracy}
\label{sec:quantitative}

In this experiment we compare the accuracy of our predictions to the
state of the art in type error localization.

\paragraph{Baseline}
We provide two baselines for the comparison: a random choice of location
from the minimized type error slice, and the standard \ocaml compiler.

\paragraph{State of the Art}
\mycroft~\citep{Loncaric2016-uk} localizes type errors by searching for
a minimal subset of typing constraints that can be removed, such that
the resulting system is satisfiable.
%
When multiple such subsets exist it can enumerate them, though it has no
notion of which subsets are \emph{more likely} to be correct, and thus
the order is arbitrary.
%
\sherrloc~\citep{Zhang2014-lv} localizes errors by searching the typing
constraint graph for constraints that participate in many unsatisfiable
paths and few satisfiable paths.
%
It can also enumerate multiple predictions, in descending order of
likelihood.

Comparing source locations from multiple tools with their own parsers is
not trivial.
%
To ensure a fair comparison when evaluating \mycroft and
\sherrloc, we removed from the dataset programs where they predicted
locations that our oracle could not match with a program expression ---
6--8\% of programs for \mycroft and 3--4\% for \sherrloc.
%
We also do not consider programs where \mycroft or \sherrloc timed out
(after one minute) or where they encountered an unsupported language
feature --- another 5\% for \mycroft and 12--13\% for \sherrloc. This
experimental design gives the state of the art tools the ``benefit of the
doubt''.


\paragraph{Our Classifiers}
We evaluate five classifiers, each trained on the full set of features.
% features: 44 local syntactic features, 176 contextual syntactic
% features, 55 typing features, and a single expression size feature.
% %
% \ES{should explain the make-up of these groups}
%
% We preemptively discard expressions that are not part of the minimal
% type error slice --- we will explain the rationale for this in
% \autoref{sec:feature-utility} --- and thus the final feature count is
% 276.
%
Our classifiers are:
%
\begin{description}
\item[\linear] A logistic regression trained with a learning rate
  $\eta = 0.001$, an $L_2$ regularization rate $\lambda = 0.001$, and a
  mini-batch size of 200.
\item[\dectree] A decision tree trained with the CART algorithm
  \citep{Breiman1984-qy} and an impurity threshold of $10^{-7}$ (used to
  avoid overfitting via early stopping).
\item[\forest] A random forest \citep{Breiman2001-wo} of 30
  estimators, trained with an impurity threshold of $10^{-7}$.
\item[\hiddenT and \hiddenFH] Two multi-layer perceptron neural
  networks, both trained with $\eta = 0.001$, $\lambda = 0.001$, and a
  mini-batch size of 200. The first MLP contains a single hidden layer
  of 10 neurons, and the second contains a hidden layer of 500
  neurons. This allows us to investigate how well the MLP can
  \emph{compress} its model (cf.~\cite{FIXME}). The neurons use
  rectified linear units (ReLU) as their activation function, a common
  practice in modern neural networks.
\end{description}
%
All classifiers were trained for 20 epochs on one dataset
--- \ie they were shown each program 20 times ---
before being evaluated on the other.
%
The logistic regression and MLPs were trained with the \textsc{Adam}
optimizer \citep{Kingma2014-ng}, a variant of stochastic gradient
descent that has been found to converge faster.


% colors from http://colorbrewer2.org/?type=sequential&scheme=Blues&n=3
\definecolor{blue1}{HTML}{DEEBF7}
\definecolor{blue2}{HTML}{9ECAE1}
\definecolor{blue3}{HTML}{3182BD}
\definecolor{green1}{HTML}{E5F5E0}
\definecolor{green2}{HTML}{A1D99B}
\definecolor{green3}{HTML}{31A354}

% \begin{figure}[ht]
% \centering
% \begin{tikzpicture}
% \begin{axis}[
%   % ybar stacked,
%   width=12cm,
%   height=8cm,
%   title={Impact of Feature Set on Accuracy},
%   ylabel={Accuracy},
%   %ymin=0.2,
%   ymax=1,
%   yticklabel={\pgfmathparse{\tick*100}\pgfmathprintnumber{\pgfmathresult}\,\%},
%   ytick style={draw=none},
%   ymajorgrids = true,
%   symbolic x coords={op+type+size, op+context+type+size, op+context-has+type+size, op+context-count+type+size},
%   % enlarge x limits=0.25,
%   xtick=data,
%   xtick style={draw=none},
%   xticklabels={Type, Context-Is, Context-Has, Context-Count},
%   x tick label style={rotate=45},
%   reverse legend,
%   transpose legend,
%   legend style={legend pos = outer north east, legend columns=4},
% ]
% % \addplot[draw=black, fill=blue1] table[x=tool, y=top-1] {\HiddenBench};
% % \addplot[draw=black, fill=blue2] table[x=tool, y expr=\thisrow{top-2} - \thisrow{top-1}] {\HiddenBench};
% % \addplot[draw=black, fill=blue3] table[x=tool, y expr=\thisrow{top-3} - \thisrow{top-2}] {\HiddenBench};

% \addplot[mark options={fill=blue1, scale=1.5}, mark=square*]
%   table[x=features, y=top-1] {\FeatureHiddenFHBench};
% \addplot[mark options={fill=blue2, scale=1.5}, mark=square*]
%   table[x=features, y=top-2] {\FeatureHiddenFHBench};
% \addplot[mark options={fill=blue3, scale=1.5}, mark=square*]
%   table[x=features, y=top-3] {\FeatureHiddenFHBench};
% \addlegendentry{Top-1}
% \addlegendentry{Top-2}
% \addlegendentry{Top-3}
% \addlegendimage{empty legend}
% \addlegendentry{\hiddenFH}

% \addplot[mark options={fill=blue1, scale=1.5}, mark=*]
%   table[x=features, y=top-1] {\FeatureLinearBench};
% \addplot[mark options={fill=blue2, scale=1.5}, mark=*]
%   table[x=features, y=top-2] {\FeatureLinearBench};
% \addplot[mark options={fill=blue3, scale=1.5}, mark=*]
%   table[x=features, y=top-3] {\FeatureLinearBench};
% \addlegendentry{Top-1}
% \addlegendentry{Top-2}
% \addlegendentry{Top-3}
% \addlegendimage{empty legend}
% \addlegendentry{\linear}

% \end{axis}
% \end{tikzpicture}
% \caption{reuslts!}
% \label{fig:results}
% \end{figure}

% \begin{figure}[ht]
% \centering
% \begin{tikzpicture}
% \begin{axis}[
%   ybar stacked,
%   width=12cm,
%   height=8cm,
%   title={Impact of Hidden Layer Size on Accuracy},
%   ylabel={Accuracy},
%   bar width=20pt,
%   %ymin=0.2,
%   ymax=1,
%   yticklabel={\pgfmathparse{\tick*100}\pgfmathprintnumber{\pgfmathresult}\,\%},
%   ytick style={draw=none},
%   ymajorgrids = true,
%   symbolic x coords={op+type+size/hidden-10, op+type+size/hidden-25, op+type+size/hidden-50,
%                      op+type+size/hidden-100, op+type+size/hidden-250, op+type+size/hidden-500},
%   % enlarge x limits=0.25,
%   xtick=data,
%   xtick style={draw=none},
%   xticklabels={\hiddenT, \hiddenTF, \hiddenF, \hiddenH, \hiddenTHF, \hiddenFH},
%   x tick label style={rotate=45},
%   reverse legend,
%   legend style={legend pos = north west},
% ]
% \addplot[draw=black, fill=blue1] table[x=tool, y=top-1] {\HiddenBench};
% \addplot[draw=black, fill=blue2] table[x=tool, y expr=\thisrow{top-2} - \thisrow{top-1}] {\HiddenBench};
% \addplot[draw=black, fill=blue3] table[x=tool, y expr=\thisrow{top-3} - \thisrow{top-2}] {\HiddenBench};
% % \addplot[draw=black, fill=blue1] table[x=tool, y=top-1] {\HiddenBench};
% % \addplot[draw=black, fill=blue2] table[x=tool, y=top-2] {\HiddenBench};
% % \addplot[draw=black, fill=blue3] table[x=tool, y=top-3] {\HiddenBench};
% \legend{Top-1, Top-2, Top-3}
% \end{axis}
% \end{tikzpicture}
% \caption{reuslts!}
% \label{fig:results}
% \end{figure}

\begin{figure}[t]
\centering
\begin{tikzpicture}
\begin{axis}[
  ybar stacked,
  width=\linewidth,
  height=7cm,
  title={Accuracy of Type Error Localization Techniques},
  ylabel={Accuracy},
  bar width=0.5cm,
  ymin=0,
  ymax=1,
  ytick={0.0, 0.1, 0.2, 0.3, 0.4, 0.5, 0.6, 0.7, 0.8, 0.9, 1.0},
  yticklabel={\pgfmathparse{\tick*100}\pgfmathprintnumber{\pgfmathresult}\,\%},
  ytick style={draw=none},
  ymajorgrids = true,
  symbolic x coords={baseline, ocaml, mycroft, sherrloc,
                     op+context+type+size/linear,
                     op+context+type+size/decision-tree,
                     op+context+type+size/random-forest,
                     op+context+type+size/hidden-10,
                     op+context+type+size/hidden-500},
  enlarge x limits=0.07,
  xtick=data,
  xtick style={draw=none},
  xticklabels={\random, \ocaml, \mycroft, \sherrloc,
               \linear, \dectree, \forest, \hiddenT, \hiddenFH},
  %x tick label style={rotate=45, anchor=north east},
  x tick label style={font=\small},
  y tick label style={font=\small},
  reverse legend,
  transpose legend,
  legend style={legend pos = north west, legend columns=4, font=\small},
]

% ES: NOTE: ORDER OF PLOTS/LEGEND ENTRIES MATTERS

\addplot[draw=black, fill=green1, bar shift=.25cm] table[x=tool, y=top-1] {\FallBench};
\addlegendentry{Top-1}
\addplot[draw=black, fill=green2, bar shift=.25cm] table[x=tool, y expr=\thisrow{top-2} - \thisrow{top-1}] {\FallBench};
\addlegendentry{Top-2}
\addplot[draw=black, fill=green3, bar shift=.25cm] table[x=tool, y expr=\thisrow{top-3} - \thisrow{top-2}] {\FallBench};
\addlegendentry{Top-3}
\addlegendimage{empty legend}
\addlegendentry{\FALL}

\resetstackedplots

\addplot[draw=black, fill=blue1, bar shift=-.25cm] table[x=tool, y=top-1] {\SpringBench};
\addlegendentry{Top-1}
\addplot[draw=black, fill=blue2, bar shift=-.25cm] table[x=tool, y expr=\thisrow{top-2} - \thisrow{top-1}] {\SpringBench};
\addlegendentry{Top-2}
\addplot[draw=black, fill=blue3, bar shift=-.25cm] table[x=tool, y expr=\thisrow{top-3} - \thisrow{top-2}] {\SpringBench};
\addlegendentry{Top-3}
\addlegendimage{empty legend}
\addlegendentry{\SPRING}


%\legend{Top-1, Top-2, Top-3}
\end{axis}
\end{tikzpicture}
\caption{
  %
  Results of our comparison of type error localization
  techniques.
  %
  We evaluate all techniques separately on two cohorts of
  students from different instances of an undergraduate
  Programming Languages course.
  %
  Our classifiers were trained on one cohort and evaluated on the other.
  %
  All of our classifiers outperform the state-of-the-art techniques
  \mycroft and \sherrloc.%  by a 10--15\% margin in Top-1 accuracy (with
%   the exception of \linear which is only slightly better than \sherrloc).
%
}
\label{fig:accuracy-results}
\end{figure}


\paragraph{Results}
\autoref{fig:accuracy-results} shows the results of our experiment.
%
Localizing the type errors in our benchmarks amounted, on average, to
selecting one of 3 correct locations out of a slice of 10.
%
Our baseline of selecting at random achieves 30\% Top-1
accuracy (58\% Top-3), while \ocaml achieves a Top-1 accuracy of 45\%.
%
Interestingly, one only needs two \emph{random} guesses to outperform
\ocaml, with 47\% accuracy.
%
\sherrloc outperforms both baselines with 56\% Top-1 accuracy (84--86\% Top
3), while \mycroft actually underperforms \ocaml with 38--41\% Top-1
accuracy.
%
Finally, we find that \emph{all} of our classifiers outperform \sherrloc,
ranging from 58--62\% Top-1 accuracy (86--88\% Top-3) for the \linear
classifier to 71--74\% Top-1 accuracy (91\% Top-3) for the \hiddenFH.

Surprisingly, there is little variation in accuracy between classifiers.
With the exception of the \linear model, they all achieve around 70\%
Top-1 accuracy and around 90\% Top-3 accuracy.
%
This suggests that the model they learn is relatively simple.
%
In particular, notice that although the \hiddenT has $50\times$ \emph{fewer}
hidden neurons than the \hiddenFH, it only loses around 2\% accuracy.
% In particular, notice that the \hiddenT only loses around 2\% accuracy
% compared to the \hiddenFH,
%
We also note that our classifiers consistently perform better when
trained on the \FALL programs and tested on the \SPRING programs than
vice versa.
% , they appear to generalize better from the \FALL data.
% FIXME: Why? What is your explanation for this? Is it just sizes of those
% datasets or something qualitative about the program pairs in them?

\subsection{Feature Utility}
\label{sec:feature-utility}
We have shown that we can train a classifier to effectively localize
type errors, but which of the 276 features that we use are contributing
the most?
%
In order to answer this question we investigate the performance of
classifiers trained on various subsets of the features.

\subsubsection{Type Error Slice}
\label{sec:type-error-slice}
The \InSlice feature should be highly predictive --- a fix must change
at least one expression in the type-error slice.
%
Thus, our first experiment seeks to quantify the impact of \InSlice by
comparing the accuracy of a linear model on three sets of features:
%
\begin{enumerate}
\item A baseline with only local syntactic features.
\item The features of (1) extended with \InSlice.
\item The same features as (1), but we preemptively discard samples
  where \InSlice is \emph{disabled}.
\end{enumerate}
%
The key difference between (2) and (3) is that a classifier for (2) must
\emph{learn} that \InSlice is a strong predictor.
%
In contrast, a classifier for (3) must only learn about the syntactic
features, the decision to discard samples where \InSlice is disabled has
already been made by a human.
%
This has a few additional advantages: it reduces the set of candidate
locations by a factor of 7 on average, and it guarantees that any
prediction made by the classifier can fix the type error.
%
We expect that (2) will perform better than (1) as it contains more
information, and that (3) will perform better than (2) as the classifier
does not have to learn the importance of \InSlice.

We tested our hypothesis with a linear model cross-validated ($k=10$)
over the combined SP14/FA15 dataset. We used a learning rate
$\alpha=0.001$, L2 regularization rate $\lambda=0.001$, and mini-batch
size of 200. We trained for a single epoch on feature sets (1) and
(2), and for 8 epochs on (3), so that the total number of training samples
would be roughly equal for each feature set.
\ES{introduce the MLP too}
\ES{use MLP-500 instead?}
\begin{table}[ht]
  \centering
  \begin{tabular}{lrcrrrrcrrrr}
    \toprule
                       &             & & \multicolumn{4}{c} \linear        & & \multicolumn{4}{c} \hiddenF       \\
                                         \cmidrule{4-7}                        \cmidrule{9-12}
    Feature Set        & \# Features & & Top-1  & Top-2  & Top-3  & Recall & & Top-1  & Top-2  & Top-3  & Recall \\
    \midrule
    Local Syntax       & 47          & & 23.6\% & 42.6\% & 56.3\% & 19.6\% & & 30.9\% & 47.9\% & 58.6\% & 20.6\% \\
    + \InSlice         & 48          & & 46.4\% & 65.0\% & 75.4\% & 30.4\% & & 54.5\% & 70.5\% & 81.5\% & 34.7\% \\
    Filter by \InSlice & 47          & & 54.9\% & 71.4\% & 82.5\% & 57.6\% & & 56.7\% & 72.3\% & 82.9\% & 58.0\% \\
    \bottomrule
  \end{tabular}
  \caption{
    Impact of type-error slice on accuracy.
    \ES{TODO: expand caption}
    \ES{TODO: load these numbers from CSV}
  }\label{tab:type-error-slice}
\end{table}

\autoref{tab:type-error-slice} shows the results of our experiment.
%
As expected the baseline performs the worst, with a mere 23.6\% Top-1
accuracy.
%
Adding \InSlice improves the results substantially with a 46.4\% Top-1
accuracy, demonstrating the importance of a minimal error slice.
%
However, filtering out expressions that are not part of the slice
\emph{further} improves the results to 54.9\% Top-1 accuracy.
%
Clearly, some decisions are too important to be left to a machine.

Note also the jump in Recall when we filter out expressions that are not
part of the error slice.
%
Reducing the search space not only improves our chances of making a
single correct prediction, it also allows us to make \emph{multiple}
correct predictions per program.

Given the decisive benefits of filtering out expressions that do not
belong to the type-error slice, we choose to filter all programs by
\InSlice.

\subsubsection{Contextual Features}
\label{sec:contextual-features}

Next, we will investigate the relative impact of the other three classes
of features discussed in \autoref{sec:features}.
%
For this we consider again a baseline of only local syntactic features,
extended by each combination of
%
(1) expression size,
(2) contextual syntactic features, and
(3) typing features.
%
As before we perform a 10-fold cross-validation with $\alpha = 0.001$,
$\lambda = 0.001$, and a mini-batch size of 200, but we
train for a full 20 epochs.

\begin{table}[ht]
  \centering
  \begin{tabular}{lrcrrrrcrrrr}
    \toprule
                             &             & & \multicolumn{4}{c} \linear        & & \multicolumn{4}{c} \hiddenFH      \\
                                               \cmidrule{4-7}                        \cmidrule{9-12}
    Feature Set              & \# Features & & Top-1  & Top-2  & Top-3  & Recall & & Top-1  & Top-2  & Top-3  & Recall \\
    \midrule
    Local Syntax             &  44         & & 55.0\% & 71.6\% & 82.6\% & 57.7\% & & 56.3\% & 72.1\% & 82.8\% & 57.6\% \\
    \midrule
    + Size                   &  45         & & 55.8\% & 72.8\% & 82.5\% & 57.3\% & & 60.5\% & 75.0\% & 83.8\% & 57.9\% \\
    + Context                & 220         & & 59.9\% & 77.7\% & 86.4\% & 63.0\% & & 71.4\% & 84.1\% & 90.8\% & 69.5\% \\
    + Types                  & 102         & & 62.2\% & 77.7\% & 85.7\% & 62.2\% & & 73.2\% & 84.8\% & 90.2\% & 69.2\% \\
    \midrule
    + Context + Size         & 221         & & 60.2\% & 77.9\% & 86.0\% & 62.5\% & & 71.7\% & 83.5\% & 90.7\% & 69.3\% \\
    + Types + Size           & 103         & & 62.0\% & 78.2\% & 85.6\% & 62.3\% & & 73.5\% & 85.8\% & 91.4\% & 70.5\% \\
    + Context + Types        & 275         & & 63.2\% & 80.3\% & 87.9\% & 65.4\% & & 77.3\% & 87.5\% & 92.4\% & 72.9\% \\
    \midrule
    + Context + Types + Size & 276         & & 62.3\% & 80.0\% & 88.0\% & 65.4\% & & 77.2\% & 87.9\% & 92.7\% & 72.9\% \\
    \bottomrule
  \end{tabular}
  % \begin{minipage}{0.49\linewidth}
  % \centering
  % \hiddenF
  % \begin{tabular}{lrrrr}
  %   \toprule
  %   Feature Set                 & Top-1  & Top-2  & Top-3  & Recall \\
  %   \midrule
  %   Local Syntax                & 56.9\% & 72.2\% & 82.8\% & 57.9\% \\
  %   \midrule
  %   + Size                      & 59.7\% & 74.6\% & 83.0\% & 57.4\% \\
  %   + Context                   & 70.9\% & 83.7\% & 90.4\% & 69.2\% \\
  %   + Types                     & 72.1\% & 84.1\% & 90.3\% & 69.3\% \\
  %   \midrule
  %   + Size + Context            & 69.8\% & 83.5\% & 90.2\% & 68.6\% \\
  %   + Size + Types              & 72.3\% & 84.6\% & 90.3\% & 69.5\% \\
  %   + Context + Types           & 75.5\% & 86.4\% & 91.5\% & 71.7\% \\
  %   \midrule
  %   + All                       & 75.0\% & 86.8\% & 91.9\% & 72.0\% \\
  %   \bottomrule
  % \end{tabular}
  % \end{minipage}
  \caption{
    Impact of contextual features on accuracy.
    \ES{TODO: expand caption}
    \ES{TODO: load these numbers from CSV}
  }\label{tab:contextual-features}
\end{table}

\paragraph{Results}
\autoref{tab:contextual-features} summarizes the results of our experiment.
%
As we can see, the \linear classifier and the \hiddenFH start off
competitive when given only local syntactic features, but the \hiddenFH
quickly begins to outperform as more features are added.

\ExprSize appears to be the weakest feature, improving \linear Top-1
accuracy by only 1\% and \hiddenFH by only 4\%.
%
In contrast, the contextual syntactic features improve \linear Top-1
accuracy by 5\% (\resp 15\%), and the typing features improve
Top-1 accuracy by 7\% (\resp 17\%).
%
Furthermore, while \ExprSize does provide some benefit when it is the
only additional feature, it does not appear to provide any real increase
in accuracy when added alongside the contextual or typing features.

As one might expect, the typing features are more beneficial than the
contextual syntactic features.
%
They improve Top-1 accuracy by an additional 2\%, and are much more
compact --- we only have 55 typing features compared to 176 contextual
syntactic features.
%
This aligns with our intuition that types should be a good summary of
the context of an expression.
%
However, typing features do not \emph{subsume} contextual syntactic
features, we can gain an additional 1\% Top-1 accuracy (\resp 4\%) by
adding \emph{both}.



%%% Local Variables:
%%% mode: latex
%%% TeX-master: "main"
%%% End:
